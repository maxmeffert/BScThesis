\documentclass[runningheads,a4paper]{llncs}

\usepackage[utf8]{inputenc}
\usepackage[T1]{fontenc}
%\usepackage[ngerman]{babel}
\usepackage{enumerate}
\usepackage{marvosym}
\usepackage{gensymb}

\usepackage[bookmarks,bookmarksopen,bookmarksdepth=2]{hyperref}

\usepackage{amsmath}
\usepackage{amsfonts}
\usepackage{amssymb}
%\usepackage{amsthm}

\usepackage{mathrsfs}

\usepackage{fancyhdr}

%\newtheorem{definition}{Definition}
%\newtheorem{theorem}{Theorem}
\newtheorem{axiom}{Axiom}


%opening
\title{BSc Notes}
\author{Maximilian Meffert}


\newcommand{\setname}[1]{\text{\textsf{#1}}}

\newcommand{\partOf}{\text{\textsf{partOf}}}
\newcommand{\partOfPlus}{\partOf^+}
\newcommand{\partOfStar}{\partOf^*}


\begin{document}

\maketitle

\begin{abstract}
asdf
\end{abstract}

%\tableofcontents

\section{Theoretical Background}

\subsection{Relations \& Predicates}

\subsubsection{Notation}
For a binary relation $R \subseteq X \times Y$ we consider the following notations to be semantically equivalent:
\begin{align*}
R(x,y)
= xRy
:= (x,y) \in R
\end{align*}
For that reason, relationships and predicates should be thought of as interchangeable. 


\subsubsection{Composition of Relations}
We define the composition of two binary relations $R \subseteq X \times Y$ and $S \subseteq Y \times X$ as:
\begin{align*}
S \circ R = \{ (x,y) | \exists z \in Y : xRz \wedge zSy \}
\end{align*}

\subsubsection{Transitive Closures of Relations: $R^+, R^*$}
For a binary relation $R$ on a set $X$ ($R \subseteq X \times X$) we define its \textit{transitive closures} $R^+$ and $R^*$ as
\begin{align*}
R^0 &:= \{ (x,x) | x \in X \} \\
R^1 &:= R \\
R^{n+1} &:= R \circ R^n \\
R^+ &:= \bigcup\limits_{1 \leq n} R^n \\
R^* &:= \bigcup\limits_{0 \leq n} R^n = R^0 \cup R^+
\end{align*}

\subsection{Mereologoy}



\subsubsection{Parthood}
\cite{DBLP:journals/dke/Varzi96}
\begin{definition}[Parthood]
The binary relation $\partOf$ and defined:
\begin{align*}
\partOf \subseteq X \times Y
\end{align*}
and literally means that one element of set $X$ may be component or \textit{part} of an element of set $Y$.
We do not make any assumption of the relationship between both sets.
However, we assume $\partOf$ to be ...
\begin{enumerate}


\item
\textbf{reflexive}
$\forall x \in X, : \partOf(x,x)$

\item
\textbf{antysymmetric}
$\forall (x,y) \in X \times Y : \partOf(x,y) \wedge \partOf(y,x) \Rightarrow x = y$

%\item 
%\textbf{asymmetric}
%$\forall x,y \in L : \partOf(x,y) \Rightarrow \neg (y \partOf x)$

\item
\textbf{transitive}
$\forall x,y,z \in L : \partOf(x,y) \wedge \partOf(y,z) \Rightarrow \partOf(x,z)$

\end{enumerate}

\end{definition}


\begin{axiom}[M4]
content...
\end{axiom}


\begin{corollary}[Transitive Closures of PartOf]
For the transitve closures
\begin{align*}
\partOfPlus, \partOfStar \in L \times L
\end{align*}
of $\partOf$ holds:
\begin{enumerate}

\item 
$\partOfPlus \subset \partOfStar$

\item
$\forall x,y \in L : x \neq y \wedge \partOf(x,y) \Rightarrow \partOfPlus(x,y)$

\item
$\forall x,y \in L : \partOfPlus(x,y) \Rightarrow \partOfStar(x,y)$

\item
$\forall x \in L :\partOfStar(x,x)$

\end{enumerate}
%\begin{align*}
%\forall x,y \in L : x \partOf y \Rightarrow x \partOfPlus y \Rightarrow x \partOfStar y
%\end{align*}
\end{corollary}

%\begin{align*}
%\forall x,y \in L : (x,y) \in \partOf
%&\Rightarrow (x,y) \in \partOf^1 \\
%&\Rightarrow (x,y) \in (\bigcup\limits_{1 \leq n} \partOf^n) \\
%&\Rightarrow (x,y) \in \partOfPlus \\
%&\Rightarrow true \wedge (x,y) \in \partOfPlus \\
%&\Rightarrow (x,x) \in \partOf \wedge (x,y) \in \partOf \\
%&\Rightarrow (x,y) \in \partOf^0 \circ \partOf^1
%\end{align*}

\subsubsection{Proper Parts, Overlaps \& Underlaps}

\begin{definition}[Proper Part]

\begin{align*}
\forall x \in X, y \in Y: \text{\textsf{properPartOf}}(x,y) :\Leftrightarrow \partOf(x,y) \wedge \neg \partOf(y,x)
\end{align*}

\end{definition}

\newcommand{\overlap}{\text{\textsf{overlaps}}}
\newcommand{\underlap}{\text{\textsf{underlaps}}}

\begin{definition}[Overlap]

\begin{align*}
\forall x \in X, y \in Y: \text{\textsf{overlaps}}(x,y) :\Leftrightarrow \exists z (\partOf(z,x) \wedge \partOf(z,y))
\end{align*}

\end{definition}

\begin{definition}[Underlap]

\begin{align*}
\forall x \in X, y \in Y: \text{\textsf{underlaps}}(x,y) :\Leftrightarrow \exists z (\partOf(x,z) \wedge \partOf(y,z))
\end{align*}

\end{definition}


\begin{definition}[Sum]

\begin{align*}
\underlap(x,y) 
&\Rightarrow \exists u \forall v (\overlap(v,u) \Leftrightarrow (\overlap(v,x) \vee \overlap(v,y))) \\
&\Rightarrow \exists u \forall v (\overlap(v,u) \Leftrightarrow (\overlap(v,x) \vee \overlap(v,y))) \\
\end{align*}

\end{definition}

\subsection{Mereotopologoy}


\subsection{Grammars \& Languages}

\section{A Mereologoy for Languages}

%===========================================================================

\section{Weblinks}

\newcommand{\urlitem}[1]{\item\url{#1}}

\begin{enumerate}
\urlitem{https://plato.stanford.edu/entries/mereology/}
\urlitem{https://de.wikipedia.org/wiki/Hasse-Diagramm}
\urlitem{http://ontology.buffalo.edu/smith/articles/mereotopology.htm}
\urlitem{https://en.wikipedia.org/wiki/Ontology_(information_science)}
\urlitem{https://en.wikipedia.org/wiki/Mereology}
\urlitem{https://en.wikipedia.org/wiki/Mereotopology}
\urlitem{https://en.wikipedia.org/wiki/Meronomy}
\urlitem{https://en.wikipedia.org/wiki/Connected_space}
\urlitem{https://en.wikipedia.org/wiki/First-order_logic} % Many-sorted_logic
\urlitem{https://userpages.uni-koblenz.de/~softlang/lao/}
\urlitem{https://de.wikipedia.org/wiki/Korrespondenz_(Mathematik)}
\urlitem{http://www.theoretische-informatik.com/relationen.php}
\urlitem{https://en.wikipedia.org/wiki/Natural_deduction}
\urlitem{https://en.wikipedia.org/wiki/Order_theory}
\urlitem{https://de.wikipedia.org/wiki/Ordnungsrelation}
\urlitem{https://en.oxforddictionaries.com/definition/meronym}
\urlitem{https://www.w3.org/TR/rdf-primer/}
\urlitem{https://www.w3.org/2001/sw/BestPractices/OEP/SimplePartWhole/}
\urlitem{https://www.w3.org/2004/02/skos/}
\urlitem{https://plato.stanford.edu/entries/set-theory/ZF.html}
\urlitem{https://de.wikipedia.org/wiki/Zermelo-Fraenkel-Mengenlehre}
\urlitem{https://en.wikibooks.org/wiki/Set_Theory/Zermelo-Fraenkel_Axiomatic_Set_Theory}
\urlitem{https://en.wikibooks.org/wiki/Set_Theory/Sets}
\urlitem{https://en.wikibooks.org/wiki/Set_Theory/Relations}
\urlitem{https://de.wikipedia.org/wiki/Relation_(Mathematik)}
\urlitem{https://de.wikipedia.org/wiki/Ordnungsrelation}
\urlitem{https://de.wikibooks.org/wiki/Mathematik:_Analysis:_Grundlagen:_Relationen}
\urlitem{https://en.wikipedia.org/wiki/Fragment_(logic)}
\end{enumerate}



%===========================================================================

\bibliographystyle{splncs}
\bibliography{notes}{}

\end{document}

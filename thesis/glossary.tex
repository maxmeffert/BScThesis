\newglossaryentry{XML}
{
    name=XML,
    description={Extensible Markup Language},
    first={XML (Extensible Markup Language)},
    text={XML}
}

\newglossaryentry{XSD}
{
    name=XSD,
    description={XML Schema Definition},
    first={XSD (XML Schema Definition)},
    text={XSD}
}

\newglossaryentry{JSON}
{
    name=JSON,
    description={JavaScript Object Notation},
    first={JSON (JavaScript Object Notation)},
    text={JSON}
}

\newglossaryentry{UML}
{
    name=UML,
    description={Unified Modeling Language},
    first={UML (Unified Modeling Language)},
    text={UML}
}

\newglossaryentry{O/R-Mapping}
{
    name=O/\-R-Mapping,
    description={Object-Relational-Mapping. The bidirectional mapping between objects in the sense of object-oriented programming and a relational database system. It usually maps classes to tables and instances to rows of a table},
    first={O/R-Mapping (Object-Relational-Mapping)},
    text={O/\-R-Mapping}
}

\newglossaryentry{O/X-Mapping}
{
    name=O/\-X-Mapping,
    description={Object-XML-Mapping. The bidirectional mapping between objects in the sense of object-oriented programming and their \gls{XML} \glspl{Representation}. It usually maps classes to \gls{XSD} and instances to \gls{XSD} entities},
    first={O/X-Mapping (Object-XML-Mapping)},
    text={O/\-X-Mapping}
}

\newglossaryentry{O/R/X-Mapping}
{
    name=O/\-R/\-X-Mapping,
    description={Object-Relational- and XML-Mapping},
    first={O/R/X-Mapping (Object-Relational- and XML-Mapping)},
    text={O/\-R/\-X-Mapping}
}

\newglossaryentry{SQL}
{
    name=SQL,
    description={Structured Query Language},
    first={SQL (Structured Query Language)},
    text={SQL}
}

\newglossaryentry{SQL/DDL}
{
    name=SQL/\-DDL,
    description={The DDL subset of SQL},
}

\newglossaryentry{ANTLR}
{
    name=ANTLR,
    description={Another Tool For Language Recognition (see \cite{Parr:2013:DAR:2501720})},
    first={ANTLR (Another Tool For Language Recognition)},
    text={ANTLR}
}

\newglossaryentry{CFG}
{
    name=CFG,
    description={Context-Free Grammar},
    first={Context-Free Grammar (CFG)},
    text={CFG}
}

\newglossaryentry{JPA}
{
    name=JPA,
    description={Java Persistence API},
    first={JPA (Java Persistence API)},
    text={JPA}
}

\newglossaryentry{HRMS}
{
    name=HRMS,
    description={Human Resource Management System},
    first={Human Resource Management System (HRMS)},
    text={HRMS}
}

\newglossaryentry{101HRMS}
{
    name=101HRMS,
    description={101wiki\footnote{\url{https://101wiki.softlang.org/} (retrieved \formatdate{12}{11}{2017})} Human Resource Management System\footnote{\url{https://101wiki.softlang.org/101:@system} (retrieved \formatdate{12}{11}{2017})}. The model used by the 101wiki for its contributions},
    first={101wiki Human Resource Management System (101HRMS)},
    text={101HRMS},
    see=[see also]{HRMS}
}

\newglossaryentry{RDBS}
{
    name=RDBS,
    description={Relational Database System},
    first={Relational Database System (RDBS)},
    text={RDBS}
}

\newglossaryentry{DDL}
{
    name=DDL,
    description={Data Definition Language. Language or subset of a language used to describe structure and content of data},
    first={Data Definition Language (DDL)},
    text={DDL}
}

\newglossaryentry{JAXB}
{
    name=JAXB,
    description={Java Architecture for XML Binding},
    first={JAXB (Java Architecture for XML Binding)},
    text={JAXB}
}

\newglossaryentry{Java}
{
    name=Java,
    description={The Java Programming Language and Platform}
}

\newglossaryentry{Hibernate}
{
    name=Hibernate,
    description={The Hibernate \gls{ORM} Framework}
}

\newglossaryentry{ORM}
{
    name=ORM,
    description={},
    see={O/R-Mapping}
}

\newglossaryentry{MegaL}
{
    name=MegaL,
    description={The megamodeling language developed by the Softlang Team at the University of Koblenz-Landau for descriptively and prescriptively modeling linguistic architectures of software systems}
}

\newglossaryentry{MegaL/Xtext}
{
    name=MegaL/\-X\-text,
    description={The Xtext implementation and eclipse IDE integration of \gls{MegaL}}
}

\newglossaryentry{CST}
{
    name=CST,
    description={Concrete Syntax Tree: A tree data structure representing the concrete syntax of a parsed text.},
    first={Concrete Syntax Tree (CST)},
    text={CST}
}

\newglossaryentry{ParseTree}
{
    name={Parse Tree},
    description={},
    see={CST}
}

\newglossaryentry{AST}
{
    name=AST,
    description={Abstract Syntax Tree: A tree data structure representing the abstract syntax of a parsed text. This tree omits syntactic features like parentheses for grouping or semicolons for sequencing},
    first={AST (Abstract Syntax Tree)},
    text={AST},
    plural={ASTs},
    firstplural={ASTs (Abstract Syntax Trees)}
}

\newglossaryentry{DFS}
{
    name=DFS,
    description={The algorithmic concept of traversing a tree or graph data structure 'top-down' until reaching the end of a path before backtracking and traversing another path},
    first={Depth-First Search (DFS)},
    text={DFS}
}

\newglossaryentry{API}
{
    name=API,
    description={Application Programming Interface},
    first={API (Application Programming Interface)},
    text={API}
}

\newglossaryentry{IDE}
{
    name=IDE,
    description={Integrated Development Environment},
    first={IDE (Integrated Development Environment)},
    text={IDE}
}

\newglossaryentry{DTO}
{
    name=DTO,
    description={Data Transfer Object. Objects with no relevant (business) logic of their own. Their sole purpose is to carry data between layers of a software system},
    first={Data Transfer Object (DTO)},
    text={DTO}
}

\newglossaryentry{GoF}
{
    name=GoF,
    description={Gang of Four. A group of authors (Erich Gamma, Richard Helm, Ralph Johnson and John Vlissides) publishing on the subject of object-oriented software design. The term may also refer to design patterns described in their book \textit{Design Patterns: Elements of Reusable Object-Oriented Software} \cite{Gamma:1995:DPE:186897}},
    first={GoF (Gang of Four)},
    text={GoF}
}

\newglossaryentry{AbstractFactoryPattern}
{
    name={Abstract Factory Pattern},
    description={A creational \gls{GoF} pattern used in software design to decouple instantiation from usage of objects. Hides the concrete nature of created instances},
    first={Abstract Factory Pattern \cite{Gamma:1995:DPE:186897}},
    text={Abstract Factory Pattern}
}

\newglossaryentry{ObserverPattern}
{
    name={Observer Pattern},
    description={A behavioral \gls{GoF} pattern used in software design to propagate state changes from one object to many dependent objects},
    first={Observer Pattern \cite{Gamma:1995:DPE:186897}},
    text={Observer Pattern}
}

\newglossaryentry{VisitorPattern}
{
    name={Visitor Pattern},
    description={A behavioral \gls{GoF} pattern used in software design to separate behavior from structure. Visitors facilitate the extension of behavior without modifying structure. The Visitor Pattern can be used to traverse object graphs},
    first={Visitor Pattern \cite{Gamma:1995:DPE:186897}},
    text={Visitor Pattern}
}

\newglossaryentry{BuilderPattern}
{
    name={Builder Pattern},
    description={A creational \gls{GoF} pattern used in software design to prevent constructor parameters from piling up},
    first={Builder Pattern \cite{Gamma:1995:DPE:186897}},
    text={Builder Pattern}
}

\newglossaryentry{StrategyPattern}
{
    name={Strategy Pattern},
    description={A behavioral \gls{GoF} pattern used in software design to separate behavior from structure. It allows to encapsulate and reuse behavior as part of the configuration of larger constructs},
    first={Strategy Pattern \cite{Gamma:1995:DPE:186897}},
    text={Strategy Pattern}
}

\newglossaryentry{FacadePattern}
{
    name={Facade Pattern},
    description={A structural \gls{GoF} pattern used in software design to simplify the usage of complex systems or \glspl{API}. It provides single access point for such system. Such access points are called facades},
    first={Facade Pattern \cite{Gamma:1995:DPE:186897}},
    text={Facade Pattern}
}


\newglossaryentry{IteratorPattern}
{
    name={Iterator Pattern},
    description={A behavioral \gls{GoF} pattern used in software design to traverse/iterate elements of a container data structure},
    first={Iterator Pattern \cite{Gamma:1995:DPE:186897}},
    text={Iterator Pattern}
}

\newglossaryentry{Artifact}
{
    name={artifact},
    description={An object created during a software development process for a certain purpose. The term artifact usually refers to a digital document or a well-formed part of it}
}

\newglossaryentry{Parthood}
{
    name={parthood},
    description={The relation between an entity and its constituent parts}
}

\newglossaryentry{Similarity}
{
    name={similarity},
    description={The relation between two things, denoting they are similar in some way}
}

\newglossaryentry{Correspondence}
{
    name={correspondence},
    description={The relation between two \glspl{Artifact}, denoting they represent the same data or encode the same information (see §\ref{section:AxiomsOfLinguisticArchitectures}, axiom \ref{axiom:CorrespondsTo}). Usually both \glspl{Artifact} only differ syntactically}
}

\newglossaryentry{Conformance}
{
    name={conformance},
    description={The relation between two \glspl{Artifact}, denoting one defines the other like a \gls{Metamodel} defines a model (see §\ref{section:AxiomsOfLinguisticArchitectures}, axiom \ref{axiom:ConformsTo})}
}

\newglossaryentry{Mereology}
{
    name={mereology},
    description={The philosophical and logical discipline of studying the constituent parts of things and the relations in between (see \cite{DBLP:journals/dke/Varzi96} and \cite{SEP:Mereology})}
}

\newglossaryentry{Megamodel}
{
    name={megamodel},
    description={A model of models. Megamodels describe relations among different kinds of models, e.g. the relations between models and their metamodels or models and their instances}
}

\newglossaryentry{Megamodeling}
{
    name={megamodeling},
    description={The task of creating and maintaining a \gls{Megamodel}}
}

\newglossaryentry{Metamodel}
{
    name={metamodel},
    description={The syntactic and semantic specification of a model}
}

\newglossaryentry{Metamodeling}
{
    name={metamodeling},
    description={The task of creating and maintaining a \gls{Metamodel}}
}

\newglossaryentry{Fragment}
{
    name={fragment},
    description={A syntactically well-formed piece of a possibly larger text}
}

\newglossaryentry{HTTP}
{
    name=HTTP,
    description={Hypertext Transfer Protocol},
    first={HTTP (Hypertext Transfer Protocol)},
    text={HTTP}
}

\newglossaryentry{Representation}
{
    name={representation},
    description={The syntactic encoding of information or a conceptual model}
}

\newglossaryentry{Manifestation}
{
    name={manifestation},
    description={},
    see={Representation}   
}

\newglossaryentry{Language}
{
    name={language},
    description={Umbrella term for natural, formal and programming languages. In context of this thesis, the term is usually refers to the latter two, unless noted otherwise}
}

\newglossaryentry{Technology}
{
    name={technology},
    description={Programs, \glspl{Language}, \glspl{API}, etc. used in a software development process},
    plural={technologies}
}

\newglossaryentry{Ontology}
{
    name={ontology},
    description={Domain knowledge captured in form of an \gls{ERModel}},
    plural={ontologies}
}


\newglossaryentry{LinguisticArchitecture}
{
    name={linguistic architecture},
    description={The architecture or model of a software system concerned with its outline regarding \glspl{Language} and \glspl{Technology} used by it or used to build it (see  \cite{DBLP:conf/models/FavreLV12}, \cite{DBLP:conf/ecmdafa/LammelV14} and \cite{DBLP:conf/modelsward/HeinzLV17})}
}

\newglossaryentry{ERModel}
{
    name={ER Model},
    description={Entity-Relationship Model},
    first={ER Model (Entity-Relationship Model)},
    text={ER Model},
    plural={ER Models},
    firstplural={ER Models (Entity-Relationship Models)}
}

\newglossaryentry{Trace}
{
    name={trace},
    description={A \gls{TraceLink} or the act of following that link \cite{DBLP:books/daglib/p/GotelCHZEGDAMM12} (see §\ref{subsection:TraceabilityTerminology})}
}

\newglossaryentry{Traceability}
{
    name={traceability},
    description={The potential for traces to be established and used \cite{DBLP:books/daglib/p/GotelCHZEGDAMM12} (see §\ref{subsection:TraceabilityTerminology})}
}

\newglossaryentry{TraceLink}
{
    name={trace link},
    description={A semantic link between two \glspl{Artifact} \cite{DBLP:books/daglib/p/GotelCHZEGDAMM12} (see §\ref{subsection:TraceabilityTerminology})}
}

\newglossaryentry{TraceRecovery}
{
    name={trace recovery},
    description={The action of creating \glspl{Trace} among already existing \glspl{Artifact} \cite{DBLP:books/daglib/p/GotelCHZEGDAMM12} (see §\ref{subsection:TraceabilityTerminology})}
}

\newglossaryentry{TraceLinkRecovery}
{
    name={trace link recovery},
    description={},
    see={TraceRecovery}
}

\newglossaryentry{StaticProgramAnalysis}
{
    name={static program analysis},
    description={The automated analysis of programs without executing them. It is the opposite of \gls{DynamicProgramAnalysis}}
}

\newglossaryentry{DynamicProgramAnalysis}
{
    name={dynamic program analysis},
    description={The automated analysis of a program's behavior and side-effects while executing it. It is the opposite of \gls{StaticProgramAnalysis}}
}

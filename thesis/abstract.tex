%Gemäß der Prüfungsordnung ist ein Abstract in deutscher und englischer Sprache verpflichtend.

\subsection*{Zusammenfassung}
Moderne Software Systeme sind heterogen bezüglich der verwendeten Sprachen und Technologien, welche zur Programmierung und Inter\--Prozess\--Kom\-mu\-ni\-ka\-ti\-on verwendet werden, was eine Herausforderung für das kognitive Verständniss von Programmen darstellt.
Untersützung von \Gls{Traceability} (Rück\-ver\-folg\-bar\-keit) kann helfen diese Herausforderung zu meistern, indem verwandte Komponenten eines Software Systems hervorgehoben werden.
Diese Abschlussarbeit skiz\-ziert einen Ansatz zur Wiederherstllung von \Gls{Traceability} zwischen Quellcodeartefakten mittles statischer Programmanalyse.
Hierfür wird ein exemplarisches System zur Wiederhrestllung von Trace Links (Spurverküpfungen), welches Trace Links mit der Semantik von Linguistic Architectures (linguistische Architekturen) wiederherstellt, entwickelt und mit einer Mini-Fallstudie evaluiert.

\subsection*{Abstract}
Modern software system are heterogeneous in terms of \glspl{Language} and \glspl{Technology} used for programming and inter process communication, which is a challenge for program comprehension.
Support of \gls{Traceability} can provide help to overcome this challenge by bringing out related components of a software system.
This thesis outlines an approach for recovering \gls{Traceability} among source code \glspl{Artifact} by utilizing \glspl{StaticProgramAnalysis}.
For this, an exemplary system for recovering \glspl{TraceLink} with the semantics of \glspl{LinguisticArchitecture} is developed and evaluated with a mini case-study.



\documentclass[bsc,m,palatino,oneside,12pt]{softlang}

% Use one of the following terms and replace msc as argument in documentclass:
% Master of Science thesis: msc
% Bachelor of Science thesis: bsc
% Projektpraktikum: pp
% Diplomarbeit: diplom
% Proseminar: prosem
% Seminararbeit: sem
% Skript: scr
% Studienarbeit: sa

% Use m(Männlich) or w(Weiblich) as argument in documentclass
% 12pt is the size of the text
% oneside is for normal text layout, the text is centered
% twoside prints the text assymmetric
% palatino is tzhe font of the letters


% BibLatex for Bibliography
% For Citation- and Bibliographystyles refer to the biblatex documentation
% Standard is numeric both
\usepackage{biblatex}
\addbibresource{bibliography.bib}

% This are optional packages, to use them remove the %:
\usepackage{booktabs} %for optimicing tabulars
%\usepackage{bibgerm} % for german BibTex styles
\usepackage{wrapfig}% for wrapping text arount tables and figures
\usepackage{multicol} % for Intermix single and multiple columns
\usepackage{enumerate}
\usepackage[inline]{enumitem} % for more options in the enumerate environment
\usepackage{subcaption} % for subfigures
\usepackage[font=footnotesize,labelfont=bf,labelsep=space]{caption} % for more formatting captions
\usepackage{tikz} % for PGF/TikZ support
\usepackage{tikz-qtree}
\usepackage{tabularx} % for advanced tables
\usepackage{listings} % for code listings
\usepackage{multirow} % for columns spanning multiple rows in tables

\usepackage[english]{babel}
\usepackage{hyphenat}

%\usepackage{lipsum}
\usepackage{amsmath}
\usepackage{amsfonts}
\usepackage{amssymb}

\usepackage{wasysym}

\usepackage[nodayofweek]{datetime}

\usepackage[toc]{glossaries}
\makeglossaries
\newglossaryentry{XML}
{
    name=XML,
    description={Extensible Markup Language},
    first={XML (Extensible Markup Language)},
    text={XML}
}

\newglossaryentry{XSD}
{
    name=XSD,
    description={XML Schema Definition},
    first={XSD (XML Schema Definition)},
    text={XSD}
}

\newglossaryentry{JSON}
{
    name=JSON,
    description={JavaScript Object Notation},
    first={JSON (JavaScript Object Notation)},
    text={JSON}
}

\newglossaryentry{UML}
{
    name=UML,
    description={Unified Modeling Language},
    first={UML (Unified Modeling Language)},
    text={UML}
}

\newglossaryentry{O/R-Mapping}
{
    name=O/R-Mapping,
    description={Object-Relational-Mapping},
    first={O/R-Mapping (Object-Relational-Mapping)},
    text={O/R-Mapping}
}

\newglossaryentry{O/X-Mapping}
{
    name=O/X-Mapping,
    description={Object-XML-Mapping},
    first={O/X-Mapping (Object-XML-Mapping)},
    text={O/X-Mapping}
}

\newglossaryentry{O/R/X-Mapping}
{
    name=O/R/X-Mapping,
    description={Object-Relational- and XML-Mapping},
    first={O/R/X-Mapping (Object-Relational- and XML-Mapping)},
    text={O/R/X-Mapping}
}

\newglossaryentry{SQL}
{
    name=SQL,
    description={Structured Query Language},
    first={SQL (Structured Query Language)},
    text={SQL}
}

\newglossaryentry{SQL/DDL}
{
    name=SQL/DDL,
    description={The DDL subset of SQL},
}

\newglossaryentry{ANTLR}
{
    name=ANTLR,
    description={Another Tool For Language Recognition},
    first={ANTLR (Another Tool For Language Recognition)},
    text={ANTLR}
}

\newglossaryentry{CFG}
{
    name=CFG,
    description={Context-Free Grammar},
    first={Context-Free Grammar (CFG)},
    text={CFG}
}

\newglossaryentry{JPA}
{
    name=JPA,
    description={Java Persistence API},
    first={JPA (Java Persistence API)},
    text={JPA}
}

\newglossaryentry{HRMS}
{
    name=HRMS,
    description={Human Resource Management System},
    first={Human Resource Management System (HRMS)},
    text={HRMS}
}

\newglossaryentry{101HRMS}
{
    name=101HRMS,
    description={101wiki\footnote{\url{https://101wiki.softlang.org/} (retrieved \formatdate{12}{11}{2017})} Human Resource Management System\footnote{\url{https://101wiki.softlang.org/101:@system} (retrieved \formatdate{12}{11}{2017})}. The model used by the 101wiki for its contributions},
    first={101wiki Human Resource Management System (101HRMS)},
    text={101HRMS},
    see=[see also]{HRMS}
}

\newglossaryentry{RDBS}
{
    name=RDBS,
    description={Relational Database System},
    first={Relational Database System (RDBS)},
    text={RDBS}
}

\newglossaryentry{DDL}
{
    name=DDL,
    description={Data Definition Language. Language or subset of a language used to describe structure and content of data},
    first={Data Definition Language (DDL)},
    text={DDL}
}

\newglossaryentry{JAXB}
{
    name=JAXB,
    description={Java Architecture for XML Binding},
    first={JAXB (Java Architecture for XML Binding)},
    text={JAXB}
}

\newglossaryentry{Java}
{
    name=Java,
    description={The Java Programming Language and Platform}
}

\newglossaryentry{Hibernate}
{
    name=Hibernate,
    description={The Hibernate \gls{ORM} Framework}
}

\newglossaryentry{ORM}
{
    name=ORM,
    description={},
    see={O/R-Mapping}
}

\newglossaryentry{MegaL}
{
    name=MegaL,
    description={The megamodeling language developed by the Softlang Team at the University of Koblenz-Landau for descriptively and prescriptively modeling linguistic architectures of software systems}
}

\newglossaryentry{MegaL/Xtext}
{
    name=MegaL/Xtext,
    description={The Xtext implementation and eclipse IDE integration of \gls{MegaL}}
}

\newglossaryentry{CST}
{
    name=CST,
    description={Concrete Syntax Tree: A tree data structure representing the concrete syntax of a parsed text.},
    first={Concrete Syntax Tree (CST)},
    text={CST}
}

\newglossaryentry{ParseTree}
{
    name={Parse Tree},
    description={},
    see={CST}
}

\newglossaryentry{AST}
{
    name=AST,
    description={Abstract Syntax Tree: A tree data structure representing the abstract syntax of a parsed text. This tree omits syntactic features like parentheses for grouping or semicolons for sequencing},
    first={AST (Abstract Syntax Tree)},
    text={AST},
    plural={ASTs},
    firstplural={ASTs (Abstract Syntax Trees)}
}

\newglossaryentry{DFS}
{
    name=DFS,
    description={The algorithmic concept of traversing a tree or graph data structure 'top-down' until reaching the end of a path before backtracking and traversing another path},
    first={Depth-First Search (DFS)},
    text={DFS}
}

\newglossaryentry{API}
{
    name=API,
    description={Application Programming Interface},
    first={API (Application Programming Interface)},
    text={API}
}

\newglossaryentry{IDE}
{
    name=IDE,
    description={Integrated Development Environment},
    first={IDE (Integrated Development Environment)},
    text={IDE}
}

\newglossaryentry{DTO}
{
    name=DTO,
    description={Data Transfer Object. Objects with no relevant (business) logic of their own. Their sole purpose is to carry data between layers of a software system},
    first={Data Transfer Object (DTO)},
    text={DTO}
}

\newglossaryentry{GoF}
{
    name=GoF,
    description={Gang of Four. A group of authors (Erich Gamma, Richard Helm, Ralph Johnson and John Vlissides) publishing on the subject of object-oriented software design. The term may also refer to design patterns described in their book \textit{Design Patterns: Elements of Reusable Object-Oriented Software} \cite{Gamma:1995:DPE:186897}},
    first={GoF (Gang of Four)},
    text={GoF}
}

\newglossaryentry{AbstractFactoryPattern}
{
    name={Abstract Factory Pattern},
    description={A creational \gls{GoF} pattern used in software design to decouple instantiation from usage of objects. Hides the concrete nature of created instances},
    first={Abstract Factory Pattern \cite{Gamma:1995:DPE:186897}},
    text={Abstract Factory Pattern}
}

\newglossaryentry{ObserverPattern}
{
    name={Observer Pattern},
    description={A behavioral \gls{GoF} pattern used in software design to propagate state changes from one object to many dependent objects},
    first={Observer Pattern \cite{Gamma:1995:DPE:186897}},
    text={Observer Pattern}
}

\newglossaryentry{VisitorPattern}
{
    name={Visitor Pattern},
    description={A behavioral \gls{GoF} pattern used in software design to separate behavior from structure. Visitors facilitate the extension of behavior without modifying structure. The Visitor Pattern can be used to traverse object graphs},
    first={Visitor Pattern \cite{Gamma:1995:DPE:186897}},
    text={Visitor Pattern}
}

\newglossaryentry{BuilderPattern}
{
    name={Builder Pattern},
    description={A creational \gls{GoF} pattern used in software design to prevent constructor parameters from piling up},
    first={Builder Pattern \cite{Gamma:1995:DPE:186897}},
    text={Builder Pattern}
}

\newglossaryentry{StrategyPattern}
{
    name={Strategy Pattern},
    description={A behavioral \gls{GoF} pattern used in software design to separate behavior from structure. It allows to encapsulate and reuse behavior as part of the configuration of larger constructs},
    first={Strategy Pattern \cite{Gamma:1995:DPE:186897}},
    text={Strategy Pattern}
}

\newglossaryentry{FacadePattern}
{
    name={Facade Pattern},
    description={A structural \gls{GoF} pattern used in software design to simplify the usage of complex systems or \glspl{API}. It provides single access point for such system. Such access points are called facades},
    first={Facade Pattern \cite{Gamma:1995:DPE:186897}},
    text={Facade Pattern}
}

\newglossaryentry{Artifact}
{
    name={artifact},
    description={Objects created during a software development process for a certain purpose. The term artifact usually refers to a digital document or a well-formed part of it}
}

\newglossaryentry{Parthood}
{
    name={parthood},
    description={The relation between an entity and its constituent parts}
}

\newglossaryentry{Similarity}
{
    name={similarity},
    description={The relation between two things, denoting they are similar in some way}
}

\newglossaryentry{Correspondence}
{
    name={correspondence},
    description={The relation between two \glspl{Artifact}, denoting they represent the same data or encode the same information (see §\ref{section:AxiomsOfLinguisticArchitectures}, axiom \ref{axiom:CorrespondsTo}). Usually both \glspl{Artifact} only differ syntactically}
}

\newglossaryentry{Conformance}
{
    name={conformance},
    description={The relation between two \glspl{Artifact}, denoting one defines the other like a \gls{Metamodel} defines a model (see §\ref{section:AxiomsOfLinguisticArchitectures}, axiom \ref{axiom:ConformsTo})}
}

\newglossaryentry{Mereology}
{
    name={mereology},
    description={The philosophical and logical discipline of studying the constituent parts of things and the relations in between (see \cite{DBLP:journals/dke/Varzi96} and \cite{SEP:Mereology})}
}

\newglossaryentry{Megamodel}
{
    name={megamodel},
    description={}
}

\newglossaryentry{Megamodeling}
{
    name={megamodeling},
    description={}
}

\newglossaryentry{Metamodel}
{
    name={metamodel},
    description={}
}

\newglossaryentry{Metamodeling}
{
    name={metamodeling},
    description={}
}

\newglossaryentry{Fragment}
{
    name={fragment},
    description={A syntactically well-formed piece of a possibly larger text}
}

\newglossaryentry{HTTP}
{
    name=HTTP,
    description={Hypertext Transfer Protocol},
    first={HTTP (Hypertext Transfer Protocol)},
    text={HTTP}
}

\newglossaryentry{Representation}
{
    name={representation},
    description={}
}

\newglossaryentry{Manifestation}
{
    name={manifestation},
    description={},
    see={Representation}   
}

\newglossaryentry{Language}
{
    name={language},
    description={}
}

\newglossaryentry{Technology}
{
    name={technology},
    description={},
    plural={technologies}
}

\newglossaryentry{Ontology}
{
    name={ontology},
    description={},
    plural={ontologies}
}


\newglossaryentry{LinguisticArchitecture}
{
    name={linguistic architecture},
    description={}
}

\newglossaryentry{ERModel}
{
    name={ER Model},
    description={Entity-Relationship Model},
    first={ER Model (Entity-Relationship Model)},
    text={ER Model},
    plural={ER Models},
    firstplural={ER Models (Entity-Relationship Models)}
}

\newglossaryentry{Trace}
{
    name={trace},
    description={}
}

\newglossaryentry{Traceability}
{
    name={traceability},
    description={}
}

\newglossaryentry{TraceabilityRecovery}
{
    name={traceability recovery},
    description={}
}

\newglossaryentry{TraceLink}
{
    name={trace link},
    description={}
}

\newglossaryentry{StaticProgramAnalysis}
{
    name={static program analysis},
    description={}
}

\usepackage{ulem}


%\usepackage{amsthm}
\usepackage{thmtools}
\newtheorem{definition}{Definition}
\newtheorem{proposition}{Propsition}
\newtheorem{proof}{Proof}
\newtheorem{axiom}{Axiom}

% Settings for the listings package, for additional settings refer to the listings package documentation
\lstset{
	breaklines=true, % Autozeilenumbruch bei langen Codezeilen
	breakatwhitespace=true, % Erlaube Zeilenumbruch nur bei Whitespace
	numbers = left, % Zeilennummern
	tabsize = 3, % Tabulatorabstand
	frame = single, % Rahmen um Sourcecode listings
	basicstyle=\tiny
	}

% Bestimmt die Tiefe in der die Kapitel nummeriert werden
\setcounter{secnumdepth}{3}

% Bestimmt die Tiefe in der die Kapitel im Inhaltsverzeichnis erscheinen
\setcounter{tocdepth}{2}

% Set headheight to 15 pt to avoid fancyhdr warning
\setlength{\headheight}{15pt}

\author{Maximilian Meffert}

\title{Trace Link Recovery\\using Static Program Analysis}

\studiengang{Informatik}

\makeatletter
%% Set the metadata for the PDF document and the colors of the internal links.
%% All colors are set to black in order to avoid unnecessary colorfulness.
\hypersetup{
	pdftitle    = {\@title},
	pdfauthor   = {\@author},
	pdfkeywords = {keyword1,keyword2,...,keywordn}, % Add some key words if you want.
	colorlinks  = true,
	unicode     = true,
	linkcolor   = black,
	citecolor   = black,
	filecolor   = black,
	urlcolor    = black,
}
\makeatother


\erstgutachter{Prof.\ Dr.\ Ralf Lämmel}
\erstgutachterInfo{Institut für Informatik}

\zweitgutachter{Msc. Johannes Härtel}
\zweitgutachterInfo{Institut für Informatik}

% \drittgutachter{Hakan Aksu}
% \drittgutachterInfo{Institut für Informatik}


%% Beware of widows and orphans.
\clubpenalty         = 10000
\widowpenalty        = 10000
\displaywidowpenalty = 10000



%%%%%%%%%%%%%%%%%%%%%%%%%%%%%%%%%%%%%%%%%%%%%%%%%%%%%%%%%%%%

\usetikzlibrary{calc}

%%%%%%%%%%%%%%%%%%%%%%%%%%%%%%%%%%%%%%%%%%%%%%%%%%%%%%%%%%%%

%\newcommand{\citedtext}[1]{\emph{"#1"}}
%
%\newcommand{\universe}{\Omega}
%\newcommand{\powerset}{\mathcal{P}}
%\newcommand{\powersetOf}[1]{\powerset(#1)}
%\newcommand{\powersetOfUniverse}{\powersetOf{\universe}}
%
%\newcommand{\relations}{\mathcal{R}}
%\newcommand{\relationsOver}[1]{\relations(#1)}
%\newcommand{\emptyRelation}{\mathcal{O}}
%\newcommand{\allRelation}{\mathcal{A}}
%\newcommand{\idRelation}{\mathcal{I}}
%
%\newcommand{\complementOf}[1]{#1^{\complement}}
%
%
%\newcommand{\Any}{\textsf{Any}}
%\newcommand{\Def}{\textsf{Def}}
%\newcommand{\DefL}[1]{\Def_{#1}}
%

\newcommand{\Entity}{\textsf{Entity}}
\newcommand{\Artifact}{\textsf{Artifact}}
\newcommand{\File}{\textsf{File}}
\newcommand{\Folder}{\textsf{Folder}}
\newcommand{\Set}{\textsf{Set}}
\newcommand{\Language}{\textsf{Language}}
\newcommand{\Fragment}{\textsf{Fragment}}

\newcommand{\partOf}{\textsf{partOf}}
\newcommand{\properPartOf}{\textsf{properPartOf}}
\newcommand{\atomicPart}{\textsf{atomicPart}}
\newcommand{\fragmentOf}{\textsf{fragmentOf}}
\newcommand{\correspondsTo}{\textsf{correspondsTo}}
\newcommand{\conformsTo}{\textsf{conformsTo}}
\newcommand{\represents}{\textsf{represents}}
\newcommand{\manifestationOf}{\textsf{manifestationOf}}
\newcommand{\sameAs}{\textsf{sameAs}}
\newcommand{\defines}{\textsf{defines}}
\newcommand{\elementOf}{\textsf{elementOf}}

\newcommand{\megal}{~\text{MegaL}~}
\newcommand{\megalxtext}{~\text{MegaL/Xtext}~}
\newcommand{\megaltext}{~\text{MegaL/Text}~}
%\newcommand{\eclipse}{~\text{eclipse}~}


\newcommand{\ToDo}[1]{
\noindent
\bfseries
{\color{red}\underline{ToDo:}~#1}
\normalfont
}

\newenvironment{contributions}
{\begin{enumerate}[
align=left,
label=\textbf{Contribution \arabic*},
ref={\arabic*}]}
{\end{enumerate}}

\newenvironment{noncontributions}
{\begin{enumerate}[
align=left,
label=\textbf{Non-Contribution \arabic*},
ref={\arabic*}]}
{\end{enumerate}}

\newenvironment{requirements}
{\begin{enumerate}[
align=left,
label=\textbf{Requirement \arabic*},
ref={\arabic*}]}
{\end{enumerate}}

%%%%%%%%%%%%%%%%%%%%%%%%%%%%%%%%%%%%%%%%%%%%%%%%%%%%%%%%%%%%

\begin{document}
\pagenumbering{Alph}
\pagestyle{empty}

\maketitle
\cleardoublepage

%Gemäß der Prüfungsordnung ist ein Abstract in deutscher und englischer Sprache verpflichtend.

\subsection*{Zusammenfassung}
%\lipsum[1]
TBD.

\subsection*{Abstract}
%\lipsum[1]
TBD.
 % Einfügen des Abstracts
\cleardoublepage

\chapter*{Acknowledgements}
I want to express my sincere gratitude to my family, my mother, father and sister for all the support they gave me throughout my education.
\newline
\vspace*{.1in}
\newline
\noindent
I also want to sincerely thank my supervisors.
Thanks to Prof. Dr. Ralf Lämmel for providing inspirational new insights into the world of software.
Thanks to Johannes Härtel for providing constructive feedback throughout the development of this thesis.
\newline
\vspace*{.1in}
\newline
\noindent
Last but not least, I want to sincerely thank my boss, Timo Ziegler, for providing conditions allowing me to work and study in parallel - and for a gentle push to get things done.

\cleardoublepage

%\pagestyle{headings}
\pagestyle{fancy}

\fancyfoot{}% Unten nichts
%\fancyhead[RE]{\itshape\leftmark}  % Rechts auf geraden Seiten=innen
%\fancyhead[RO]{\itshape\rightmark} % Links auf ungeraden Seiten=außen
\fancyhead[R]{\thepage}        % nur rechts

\pagenumbering{roman}

\tableofcontents
\cleardoublepage

\listoftheorems
\cleardoublepage

\listoffigures   % fuer ein eventuelles Abbildungsverzeichnis
\cleardoublepage

\listoftables % Für ein Tabellenverzeichnis
\cleardoublepage

\lstlistoflistings % Für ein Verzeichnis der Listings
\cleardoublepage


\pagenumbering{arabic}


% Hier kommt jetzt der eigentliche Text der Arbeit
\chapter{Introduction}
\label{chapter:Introduction}
TBD.

\section{Contributions}
\section{Non-Contributions}
\chapter{Background}
\label{chapter:Background}
This section summarizes the necessary background topics of the thesis.
Each topic is introduced independently, interrelation is done during synthesis of hypotheses for this thesis in chapter \ref{chapter:Hypotheses}.


\section{Relations}
This section introduces the necessary aspects of mathematical relations for this thesis.
The concept of relations is a generalization of semantic dependencies between two or more mathematical objects.
This section is based on \cite{DBLP:books/sp/SchmidtS89}.

Relations are based on set-theory. We also introduce the necessary constructs of set-theory in order to clarify terminology and notation.
A set is a collection of well distinguishable mathematical objects.
Objects in a set are called elements of the set.
A set does not contain two or more identical elements.
The notation $x \in X$ denotes that $x$ is an element of the set $X$.
The symbol $\emptyset$ denotes the \emph{empty set}, which contains no elements.
The symbol $\universe$ denotes the universal set, which contains all elements.

\begin{definition}[Inclusion]
\label{definition:Inclusion}
Let $X$ and $Y$ be a sets.
$Y$ \emph{includes} $X$ if and only if:
\begin{align}
X \subset Y 
&:\Leftrightarrow \forall x [x \in X \rightarrow x \in Y]
\Leftrightarrow \forall x [x \not\in X \vee x \in Y ]
\end{align}
Then $X$ is called \emph{subset} of $Y$ and $Y$ is called \emph{superset} of $X$.
For an arbitrary set $Z \neq \emptyset$, the statement $\emptyset \subset Z$ is always true, respectively $Z \subset \emptyset$ is always false.
\end{definition}
We also define the opposite property: $Y$ does not include $X$ if and only if:
\begin{align}
X \not\subset Y
&:\Leftrightarrow \exists x [x \in X \wedge x \not\in Y]
\end{align}

\begin{definition}[Union]
Let $X$ and $Y$ be a sets.
\begin{align}
X \cup Y &:= \{ x | x \in X \vee x \in Y \} 
\end{align}
$X \cup Y$ is called \emph{union} of $X$ and $Y$.
\end{definition}

\begin{definition}[Intersection]
Let $X$ and $Y$ be a sets.
\begin{align}
X \cap Y &:= \{ x | x \in X \wedge x \in Y \} 
\end{align}
$X \cap Y$ is called \emph{intersection} of $X$ and $Y$.
\end{definition}

\begin{definition}[Power-Set]
\label{definition:PowerSet}
Let $X$ be a set, then the \emph{power-set} of $X$ is defined as:
\begin{align}
\mathcal{P}(X) := \{ Y | Y \subset X \}
\end{align}
\end{definition}

The definition of inclusion provides an order for power-sets.
So we may compare two sets $A$ and $B$ in the sense of \emph{smaller} and \emph{larger}, i.e.:
\begin{align}
A \text{ is smaller than } B 
&\Leftrightarrow A \subset B
\\
A \text{ is larger than } B  
&\Leftrightarrow B \subset A
\\
A \text{ is the smallest subset of } B
&\Leftrightarrow \forall C \in \powersetOf{B} : A \subset C
\\
A \text{ is the largest subset of } B
&\Leftrightarrow \forall C \in \powersetOf{B} : C \subset A
\end{align}


\begin{definition}[Upper \& Lower Bound]
Let $\universe$ be a universe, $X \in \powersetOfUniverse$ be a set in the universe and $A \subset \mathcal{P}(U), A \neq \emptyset,$ non-empty subsets in the universe.
\begin{align}
X \text{ is an \emph{upper bound} for } A
&:\Leftrightarrow
\forall Y \in A : Y \subset X
\\
X \text{ is a \emph{lower bound} for } A
&:\Leftrightarrow
\forall Y \in A : X \subset Y
\end{align}
We also define:
\begin{align}
\mathbf{U}_A 
&:= \{ U \in \powersetOfUniverse | \forall Y \in A : Y \subset U \}
\\
\mathbf{L}_A
&:= \{ L \in \powersetOfUniverse | \forall Y \in A : L \subset Y \}
\end{align}
as sets of all upper/lower bounds for $A$.
\end{definition}

Because our definition of upper and lower bounds is based on power-sets, existence is guaranteed:
Given an arbitrary set $S$, the $S$ and $\emptyset$ are always elements of $\powersetOf{S}$.
For each element $Y$ of a non-empty selection $A \subset \powersetOf{S}$ of the power-set, $Y \subset S$ and $\emptyset \subset Y$ holds.
So $S$ is an upper and $\emptyset$ is a lower bound for $A$.

\begin{definition}[Supremum \& Infimum]
\label{definition:SupremumAndInfimum}
Let $\universe$ be a universe, $X \in \powersetOfUniverse$ be sets in the universe and $A \subset \powersetOfUniverse, A \neq \emptyset$ a non-empty selection of sets in the universe.
If
\begin{align}
X 
= \sup A
:= \bigcup\limits_{Y \in A} Y
&:\Leftrightarrow
X \in \mathbf{U}_A \wedge \forall U \in \mathbf{U}_A : X \subset U
\\
X
= \inf A
:= \bigcap\limits_{Y \in A} Y
&:\Leftrightarrow
X \in \mathbf{L}_A \wedge \forall L \in \mathbf{L}_A : L \subset X
\end{align}
then $X$ is called \emph{supremum}/\emph{infimum} for $A$.
\end{definition}

Existence for supremum and infimum is guaranteed, because upper and lower bounds exist as shown above.
Thus, for any non-empty selection $A \subset \powersetOf{S}$ of a power-set, $\mathbf{U}_A$ and $\mathbf{L}_A$ are not empty.
So we need to proof, that $X = \bigcup\limits_{Y \in A} Y$ respectively $X = \bigcap\limits_{Y \in A} Y$ are in fact the smallest upper and the  largest lower bound.
Or in other words:
Does another bound $X' \in \mathbf{U}_A$ respectively $X' \in \mathbf{L}_A$ exist with $X' \neq X$ and $X' \subset X$ respectively $X \subset X'$?
\begin{enumerate}
\item
Supremum:
We assume $X' \in \mathbf{U}_A$ with $X' \neq X$ and $X' \subset X$ for $X = \bigcup\limits_{Y \in A} Y$ exists, then an element $x \in X$ exists, which is not element of $X'$.
Because $X$ is the union of all sets in selection $A$, $x$ must be element of at least one of its sets.
However, then $X'$ cannot include sets containing $x$.
Thus, $X'$ cannot be an upper bound for $A$ and $X = \sup A$.

\item
Infimum:
We assume $X' \in \mathbf{L}_A$ with $X' \neq X$ and $X \subset X'$ for $X = \bigcap\limits_{Y \in A} Y$ exists, then an element $x \in X'$ exists, which is not element of $X$.
Because $X$ is the intersection of all sets in selection $A$, $x$ cannot be element of at least one of its sets.
However, then $X'$ must include sets containing $x$.
Thus, $X'$ cannot be a lower bound fo $A$ and $X = \inf A$.

\end{enumerate}
Supremum and Infimum are unique for any non-empty selection of sets in a universe and can be obtained by its union respectively its intersection.
\cite{DBLP:books/sp/SchmidtS89}

\begin{definition}[Cartesian Product]
Let $U$ be a universe and $X_n \in \mathcal{P}(U)$ sets with $ i=1...n, n \in \mathbb{N}$, then:
\begin{align}
X_1 \times ... \times X_n := \{ (x_1,..., x_n) \}
\end{align} 
is called \emph{Cartesian product}.
\end{definition}

\begin{definition}[Relation]
A relation is a subset of a Cartesian product:
\begin{align}
R \subset X_1 \times ... \times X_n 
\end{align}
The relation of only two sets is called \emph{binary relation}.
Instead of writing $(x,y) \in R$ we may also use the shorter notation $xRy$.
\end{definition}

An arbitrary relation $R \subset A \times B$ is called \emph{homogeneous} if $A = B$, otherwise it is called \emph{heterogeneous}.
However, an arbitrary relation $R \subset A \times B$ is also homogeneous in the sense of
$R \subset A \times B \subset (A \cup B) \times (A \cup B)$.
For the remainder of this section we focus on homogeneous relations unless noted otherwise.

In order to clarify our notation, when we are specifically working with relations instead of ordinary sets, we use the symbols
$\sqsubset$ for inclusion,$\not\sqsubset$ for non-inclusion, $\sqcup$ for union and $\sqcap$ for intersection, i.e.:
\begin{align}
R \sqsubset S
&:\Leftrightarrow
\forall x,y [(x,y) \in R \rightarrow (x,y) \in S]
\\
R \not\sqsubset S
&:\Leftrightarrow
\exists x,y [(x,y) \in R \wedge (x,y) \not\in S]
\\
R \sqcup S
&:=
\{ (x,y) | (x,y) \in R \vee (x,y) \in S \}
\\
R \sqcap S
&:=
\{ (x,y) | (x,y) \in R \wedge (x,y) \in S \}
\end{align}
Furthermore, we use $\relationsOver{A}$ to denote the set of all homogeneous relations in set $A$ and the symbols $\emptyRelation$ and $\allRelation$ to denote the empty relation and the universal relation:
\begin{align}
\relationsOver{A} 
&:= \{ R | R \subset A \times A \}
\\
(\emptyRelation &:= \emptyset) \subset A \times A
&\Leftrightarrow
\forall R \in \mathcal{R}(A) [ \emptyRelation \sqsubset R ]
\\
(\allRelation &:= A \times A) \subset A \times A
&\Leftrightarrow
\forall R \in \mathcal{R}(A) [ R \sqsubset \allRelation ]
\end{align}

\begin{definition}[Composition of Binary Relations]
%Let $R \subset A \times B$ and $S \subset C \times D$ be two binary relations.
%Then $S \circ R \subset A \times D$ is defined
%\begin{align}
%R \circ S = RS := \{ (a,d) \in A \times D | \exists x \in B \cap C : (a,x) \in R \wedge (x,d) \in S \}
%\end{align}
Let $R,S \in \relationsOver{A}$.
Then $R \circ S \in \relationsOver{A}$ is defined
\begin{align}
R \circ S 
= RS 
:= \{ (r,s) \in A \times A | \exists x \in A : (r,x) \in R \wedge (x,s) \in S \}
\end{align}
and called \emph{composition} or \emph{multiplication} of $R$ and $S$.
Instead of writing $R \circ S$ we also write simply $RS$.
\end{definition}
In conjunction with $\sqsubset$ composition is monotone:
\begin{align}
\forall P,Q,R \in \relationsOver{A}:
 P \sqsubset Q \rightarrow R \circ P \sqsubset R \circ Q
\\
\forall P,Q,R \in \relationsOver{A}:
 P \sqsubset Q \rightarrow P \circ R \sqsubset Q \circ R  
\end{align}
\begin{itemize}
\item
($\Rightarrow$)
Following the definition of composition, it can easily be observed, if $Q$ includes $P$, all elements of $P$ are also in $Q$:
\begin{align}
&\forall x,y: (x,y) \in R \circ P
\\&\rightarrow
\exists z : (x,z) \in R \wedge (z,y) \in P 
\overset{P \sqsubset Q}{\rightarrow}
\exists z :  (x,z) \in R \wedge (z,y) \in Q
\\&\rightarrow 
(x,y) \in R \circ Q
\end{align}
An analogous deduction can be shown for the right hand side composition of $R$.

\item
($\Leftarrow$)
The opposite direction can be proven indirectly, assuming $R \circ P \sqsubset R \circ Q$ holds, but $P \sqsubset Q$ does not:
%\begin{align*}
%&\forall x,y: (x,y) \in R \circ P \rightarrow (x,y) \in R \circ Q
%\\&\Leftrightarrow
%\exists z : (x,z) \in R \wedge (z,y) \in P \vee [(x,z) \in R \]
%\overset{P \not\sqsubset Q}{\rightarrow}
%\exists z : (x,z) \in R \wedge (z,y) \in P \wedge (z,y) \not\in Q
%\end{align*}

\end{itemize}




\begin{definition}[Identity Relation]
The relation $\idRelation \in \relationsOver{A}$
\begin{align}
\idRelation &:= \{ (a,b) \in A \times A | a = b \} = \{ (a,a) | a \in A \} \subset A \times A
\end{align}
is called \emph{identity relation}.
\end{definition}

$(\relationsOver{A},\circ,\idRelation)$ is a monoid, i.e. for all relations in $\relationsOver{A}$ composition is associative and $\idRelation$ serves as it's identity element:
\begin{align}
&(Q \circ R) \circ S 
= Q \circ (R \circ S)
\\
&R \circ \idRelation 
= \idRelation \circ R = R
\end{align}
Also, $\emptyRelation$ serves as absorbing element for composition:
\begin{align}
R \circ \emptyRelation = \emptyRelation \circ R = \emptyRelation
\end{align}

Because $(\relationsOver{A},\circ,\idRelation)$ is a monoid, we can define exponentiation:

\begin{definition}[Exponentiation of Relations]
Let $R \in \relationsOver{A}$ and $n \in \mathbb{N}$.
\begin{align}
R^{0} &:= \idRelation
\\R^{n} &:= R \circ R^{n-1}
\end{align}
\end{definition}

Consider the following example:
\begin{align}
A &= \{a,b,c,d\}\\
R &= \{(a,b),(a,c),(c,d)\}\\
R^{0} &= \{(a,a),(b,b),(c,c),(d,d)\}\\
R^{1} &= R \circ R^{0} = R \circ \idRelation = \{(a,b),(a,c),(c,d) \}\\
R^{2} &= R \circ R^{1} = R \circ R = \{(a,d)\}\\
R^{3} &= R \circ R^{2} = \emptyRelation\\
R^{4} &= R^{5} = R^{6} = ... = R \circ \emptyRelation = \emptyRelation
\end{align}


%\begin{definition}[Properties of Relations]
%%Let $R \subset A \times A$ and $S \subset A \times B$ be homogeneous or arbitrary relations, they may satisfy one or more of the following properties:
%Let $R \in \relationsOver{A}$ it may satisfy one or more of the following properties:
%\begin{align}
%%\emph{bijective} 
%%&:\Leftrightarrow
%%\forall b \in B \exists! a \in A: (a,b) \in S
%%\\
%%\emph{function} 
%%&:\Leftrightarrow
%%\forall a \in A \exists! b \in B: (a,b) \in S
%%\\
%\emph{reflexive} 
%&:\Leftrightarrow
%\forall a \in A: (a,a) \in R
%\\
%\emph{irreflexive} 
%&:\Leftrightarrow
%\forall a \in A: (a,a) \not\in R
%\\
%\emph{transitive} 
%&:\Leftrightarrow
%\forall a,b,c \in A: (a,b) \in R \wedge (b,c) \in R \Rightarrow (a,c) \in R
%\\
%\emph{intransitive} 
%&:\Leftrightarrow
%\forall a,b,c \in A: (a,b) \in R \wedge (b,c) \in R \Rightarrow (a,c) \not\in R
%\\
%\emph{symmetric} 
%&:\Leftrightarrow
%\forall a,b \in A: (a,b) \in R \Rightarrow (b,a) \in R
%\\
%\emph{asymmetric} 
%&:\Leftrightarrow
%\forall a,b \in A: (a,b) \in R \Rightarrow (b,a) \not\in R
%\\
%\emph{antisymmetric} 
%&:\Leftrightarrow
%\forall a,b \in A: (a,b) \in R \wedge (b,a) \in R \Rightarrow a = b
%\end{align}
%\end{definition}

\begin{definition}[Reflexivity]
A relation $R \in \relationsOver{A}$ is called:
\begin{align}
\emph{reflexive} 
&:\Leftrightarrow
\forall a \in A: (a,a) \in R
\\
\emph{irreflexive} 
&:\Leftrightarrow
\forall a \in A: (a,a) \not\in R
\end{align}
\end{definition}

\begin{definition}[Reflexive Closure]
Let $R \in \relationsOver{A}$.
Then
\begin{align}
R^\circ
:= \inf \{ S | R \sqsubset S \wedge S \text{ reflexive } \}
= R \sqcup \idRelation
\end{align}
is called \emph{reflexive closure} of $R$.
\end{definition}

The reflexive closure of a homogeneous relation $R$ is the infimum or largest lower bound of the set $A = \{ S | R \sqsubset S \wedge S \text{ reflexive } \}$ containing all reflexive relations, which include $R$.
The smallest reflexive relation is $\idRelation$.
The smallest relation including $R$ is $R$ itself.
So, for an arbitrary relation $R' \in A$ the inclusions $R \sqsubset R'$, $\idRelation \sqsubset R'$ and $(R \sqcup \idRelation) \sqsubset R'$ hold.
Thus $R \sqcup \idRelation$ is a lower bound for $A$ and $(R \sqcup \idRelation) \sqsubset \inf A$ holds.
Vice versa, $R \sqcup \idRelation$ is an element of $A$ and any relation $R'' \sqsubset (R \sqcup \idRelation)$ does either not include $R$ or is not reflexive.
Thus $R \sqcup \idRelation$ is also the smallest relation in $A$ and $\inf A \sqsubset (R \sqcup \idRelation)$.
From $(R \sqcup \idRelation) \sqsubset \inf A$ and $\inf A \sqsubset (R \sqcup \idRelation)$ follows $\inf A = (R \sqcup \idRelation)$.
\cite{DBLP:books/sp/SchmidtS89}

\begin{definition}[Symmetry]
\label{definition:Symmetry}
A relation $R \in \relationsOver{A}$ is called:
\begin{align}
\emph{symmetric} 
&:\Leftrightarrow
\forall a,b \in A: (a,b) \in R \Rightarrow (b,a) \in R
\\
\emph{asymmetric} 
&:\Leftrightarrow
\forall a,b \in A: (a,b) \in R \Rightarrow (b,a) \not\in R
\\
\emph{antisymmetric} 
&:\Leftrightarrow
\forall a,b \in A: (a,b) \in R \wedge (b,a) \in R \Rightarrow a = b
\end{align}
\end{definition}

\begin{definition}[Transitivity]
A relation $R \in \relationsOver{A}$ is called:
\begin{align}
\emph{transitive} 
&:\Leftrightarrow
R^{2} \sqsubset R
\\&:\Leftrightarrow
\forall a,b,c \in A: (a,b) \in R \wedge (b,c) \in R \rightarrow (a,c) \in R
\\
\emph{intransitive} 
&:\Leftrightarrow
R^{2} \not\sqsubset R
\\&:\Leftrightarrow
\forall a,b,c \in A: (a,b) \in R \wedge (b,c) \in R \rightarrow (a,c) \not\in R
\end{align}
\end{definition}
The equivalent characterization of transitivity is $R^{2} \sqsubset R$ is easily obtained:
\begin{align}
&\forall a,b,c \in A: (a,b) \in R \wedge (b,c) \in R \Rightarrow (a,c) \in R
\\&\Leftrightarrow
\forall a,c \in A: (a,c) \in RR \rightarrow (a,c) \in R
\\&\Leftrightarrow
\forall a,c \in A: (a,c) \in R^{2} \rightarrow (a,c) \in R
\\&\Leftrightarrow
R^{2} \sqsubset R 
\end{align}
If $\mathcal{T}(A) := \{ R \in \relationsOver{A} | R^{2} \sqsubset R \}$ is the set over all transitive relations of set $A$, it's infimum $I := \inf \mathcal{T}(A)$ is also transitive.
Assuming it is not, at least one element exists, which is in $I^{2}$, but not in $I$.
Because $I$ is the infimum of $\mathcal{T}(A)$, all transitive relations $R$ must include it.
Thus the same element is in $R^{2}$, but not in $R$.
However, this is a contradiction, because $R$ is transitive and all elements in $R^{2}$ must be in $R$.
\begin{align}
\neg(I^{2} \sqsubset I)
&\Leftrightarrow
\forall x,y : \neg[(x,y) \in I^{2} \rightarrow (x,y) \in I]
\\&\Leftrightarrow
\forall x,y : (x,y) \in I^{2} \wedge (x,y) \not\in I
\\&\Leftrightarrow
\forall x,y \exists z : (x,z) \in I \wedge (z,y) \in I \wedge (x,y) \not\in I
\\&\Leftrightarrow
\forall x,y \exists z : (x,z) \in R \wedge (z,y) \in R \wedge (x,y) \not\in R
\\&\Leftrightarrow
\forall x,y : (x,y) \in R^{2} \wedge (x,y) \not\in R
\\&\Leftrightarrow
\forall x,y : \neg[(x,y) \in R^{2} \rightarrow (x,y) \in R]
\\&\Leftrightarrow
\neg(R^{2} \sqsubset R) 
\text{\lightning}
\end{align}


\begin{definition}[Transitive Closure]
Let $R \in \relationsOver{A}$ and $i \in \mathbb{N}$.
Then
\begin{align}
R^{+}
:= \inf \{ S | R \sqsubset S \wedge S \text{ transitive } \}
= \sup \{ R^{i} | i \geq 1 \}
\end{align}
%$R^{+} := R^{1} \cup R^{2} \cup R^{3} \cup ...$
is called \emph{transitive closure} of $R$.
\end{definition}

The transitive closure of a relation is the infimum or greatest lower bound of the set $A = \{ S | R \sqsubset S \wedge S \text{ transitive } \}$ containing all transitive relations, which include $R$.
Because $A$ contains only transitive relations, its infimum $I = \inf A$ is also transitive.

\begin{align*}
&PR \sqsubset QS
\\&\Leftrightarrow
\forall x,y:
(x,y) \in PR \rightarrow (x,y) \in QS
\\&\Leftrightarrow
\forall x,y \exists z:
(x,z) \in P \wedge (z,y) \in R \rightarrow (x,z) \in Q \wedge (z,y) \in S
\\&\Leftrightarrow
\forall x,y \exists z:
(x,z) \not\in P \vee (z,y) \not\in R
\vee 
[(x,z) \in Q \wedge (z,y) \in S]
\\&\Leftrightarrow
\forall x,y \exists z:
[(x,z) \not\in P \vee (z,y) \not\in R \vee (x,z) \in Q]
\wedge
[(x,z) \not\in P \vee (z,y) \not\in R \vee (z,y) \in S]
\\&\Leftrightarrow
\forall x,y \exists z:
[(z,y) \not\in R \vee [(x,z) \not\in P \vee (x,z) \in Q]]
\wedge
[(x,z) \not\in P \vee [(z,y) \not\in R \vee (z,y) \in S]]
\\&\Leftrightarrow
\forall x,y \exists z:
[(z,y) \not\in R \vee [(x,z) \in P \rightarrow (x,z) \in Q]]
\wedge
[(x,z) \not\in P \vee [(z,y) \in R \rightarrow (z,y) \in S]]
\\&\Leftrightarrow
\forall x,y \exists z:
[(z,y) \not\in R \vee P \sqsubset Q]
\wedge
[(x,z) \not\in P \vee R \sqsubset S]
\end{align*}

\begin{align*}
&PR \not\sqsubset QS
\\&\Leftrightarrow
\exists x,y : (x,y) \in PR \wedge (x,y) \not\in QS
\\&\Leftrightarrow
\exists x,y,z: [(x,z) \in P \wedge (z,y) \in R] \wedge \neg[(x,z) \in Q \wedge (z,y) \in S]
\\&\Leftrightarrow
\exists x,y,z: [(x,z) \in P \wedge (z,y) \in R] \wedge [(x,z) \not\in Q \vee (z,y) \not\in S]
\\&\Leftrightarrow
\exists x,y,z:
[(x,z) \in P \wedge (z,y) \in R \wedge (x,z) \not\in Q]
\vee
[(x,z) \in P \wedge (z,y) \in R \wedge (z,y) \not\in S]
\\&\Leftrightarrow
\exists x,y,z:
[P \not\sqsubset Q \wedge (z,y) \in R]
\vee
[(x,z) \in P \wedge R \not\sqsubset S]
\\&\Leftrightarrow
\forall x,y,z:
\neg([P \not\sqsubset Q \wedge (z,y) \in R]
\vee
[(x,z) \in P \wedge R \not\sqsubset S])
\\&\Leftrightarrow
\forall x,y,z:
[P \sqsubset Q \vee (z,y) \not\in R]
\wedge
[(x,z) \not\in P \vee R \sqsubset S]
\\&\Leftrightarrow
\forall x,y,z:
[[P \sqsubset Q \vee (z,y) \not\in R] \wedge (x,z) \not\in P ]
\vee
[[P \sqsubset Q \vee (z,y) \not\in R] \wedge R \sqsubset S]
\\&\Leftrightarrow
\forall x,y,z:
[P \sqsubset Q \wedge (x,z) \not\in P] 
\vee 
[(z,y) \not\in R \wedge (x,z) \not\in P]
\vee
[P \sqsubset Q \wedge R \sqsubset S]
\vee 
[(z,y) \not\in R \wedge R \sqsubset S]
\\&\Leftrightarrow
\forall x,y,z:
[P \sqsubset Q \wedge (x,z) \not\in P] 
\vee 
[(x,y) \not\in PR]
\vee
[P \sqsubset Q \wedge R \sqsubset S]
\vee 
[(z,y) \not\in R \wedge R \sqsubset S]
\end{align*}



\begin{definition}[Transitive-Reflexive Closure]
$R^{*} := R^{+} \cup I$
is called \textit{reflexive-transitive closure}
\end{definition}

\begin{definition}[Order Relation]
content...
\end{definition}


\section{Formal Languages \& Grammars}
\subsection{Context-Free Languages \& Grammars}

\section{Traceability}
\cite{Winkler:2010:STR:1861285.1861287}
%\lipsum[1]
TBD.
\subsection{Traceability Relationship}
\subsection{Traceability Link}
\subsection{Traceability Recovery}
\subsection{Traceability Exploration}

\section{Megamodeling}
%\lipsum[1]
TBD.

\subsection{\megal}
\subsubsection{\megalxtext}

\section{Ontologies}
%\lipsum[1]
TBD.

\section{Mereology}
\begin{align}
&x \partOf x 
&\qquad \text{(Reflexivity)}
\\&x \partOf y \wedge y \partOf x \Rightarrow x = y
&\qquad \text{(Antisymmetry)}
\\&x \partOf y \wedge y \partOf z \Rightarrow x \partOf z
&\qquad \text{(Transitivity)}
\end{align}
%\lipsum[1]
TBD.

\section{Program Analysis}
%\lipsum[1]
TBD.

\section{XML Data Binding}
%\lipsum[1]
TBD.


\subsection{Java Architecture for XML Binding (JAXB)}

\section{Object Relational Mapping}
%\lipsum[1]
TBD.



\subsection{Java Persistence API (JPA)}

\subsection{Hibernate}



\section{Another Tool For Language Recognition (ANTLR)}


\chapter{Related Work}
\label{chapter:RelatedWork}

\section{Linguistic Architectures}

\paragraph*{Modeling the Linguistic Architecture of Software Products}
\cite{DBLP:conf/models/FavreLV12}


\section{Trace Recovery}

\paragraph*{Information Retrieval Methods for Automated Traceability Recovery}
\cite{DeLucia2012}

\paragraph*{Using Rules for Traceability Creation} 
\cite{Zisman2012}

\paragraph*{Tracing object-oriented code into functional requirements} 
\cite{AntoniolCCDM00}

\paragraph*{Recovering Code to Documentation Links in OO Systems}
\cite{AntoniolCDLM99}

\paragraph*{Mining software repositories for traceability links}
\cite{KagdiMS07}

\paragraph*{CaCOphoNy: metamodel-driven software architecture reconstruction}
\cite{Favre2004}

\paragraph*{Combining Textual and Structural Analysis of Software Artifacts for  Traceability Link Recovery}
\cite{McMillanPR2009}

\paragraph*{Traceability Management for Impact Analysis}
\cite{DelukaFR2008}

\paragraph*{Automating traceability link recovery through classification}
\cite{Mills:2017:ATL:3106237.3121280}

\paragraph*{Recovering test-to-code traceability using slicing and textual analysis} 
\cite{Qusef:2014:RTT:2747015.2747194}

\paragraph*{Recovering traceability links in software artifact management systems using information retrieval methods}
\cite{Lucia:2007:RTL:1276933.1276934}

\paragraph*{Semantic driven program analysis}
\cite{Marcus2004}






\chapter{Design}
\label{chapter:Design}
This chapter summarizes the design of the conformance- and  recovery system 
\chapter{Implementation}
%\lipsum[1]
TBD.

\section{CFG Fragmentation with ANTLR}
Context-Free Grammar Fragmentation with ANTLR

\section{Name Correspondence Heuristic}


Heuristics are quick and "simple" methods for finding good approximate solutions for complex problems.
The Name Correspondence Heuristic determines correspondence between artifacts simply by finding similarities of names in those artifacts. 

TBD.
\chapter{Mini Case-Study}
\label{chapter:MiniCaseStudy}
This chapter provides a mini case-study evaluating the developed recovery system for this thesis.
§\ref{section:ExampleCorpus} outlines the corpus used for evaluation.
§\ref{section:Metrics} describes metrics used to evaluate the system.
§\ref{section:Results} discusses results of the case-study.

\section{Corpus}
\label{section:ExampleCorpus}
The example corpus used to develop the recovery system for this thesis consists of artifacts implementing a fictional \gls{HRMS} within an \gls{O/R/X-Mapping} scenario using \gls{Java} technologies.
The model is provided by the 101wiki\footnote{\url{https://101wiki.softlang.org/} (retrieved \formatdate{12}{11}{2017})}, where it is used for contributions.
It is implemented using plain \Gls{Java} and then mapped to plain \gls{XML}/\gls{XSD} with \gls{JAXB} and to \gls{SQL/DDL} statements using \gls{Hibernate} mapping files and/or annotations.



\subsection{The 101HRMS Model}
\gls{101HRMS}\footnote{\url{https://101wiki.softlang.org/101:@system} (retrieved \formatdate{12}{11}{2017})} provides a simple model of a company with many departments and employees.
Figure \ref{figure:101HRMSModel} shows an \gls{UML} class diagram of a variant of this model.

\begin{figure}[h!]
\begin{center}
\includegraphics[scale=.5]{images/101HRMSModel.png}
\end{center}
{
\scriptsize 
This \gls{UML} class diagram depicts the model of the \gls{101HRMS}.
It consists of simple companies with nested departments and employees mapped to the latter.
}
\caption{The 101 Human Resource Management System Model}
\label{figure:101HRMSModel}
\end{figure}

The \gls{101HRMS} model consists of companies attributed with a name.
Each company accumulates departments.
Each department is also attributed with a name, aggregates employees and has one employee acting as manager.
Departments can further be refined into sub-departments.
Each employee is attributed with a name, an age and a salary.
Each entity is also attributed with an ID.

\subsection{Linguistic Domains of the Example Corpus}
The example corpus used to develop the recovery system contains artifacts implementing the \gls{101HRMS} model generated or used by \gls{Java} technologies for \gls{O/R/X-Mapping}, i.e. a \gls{Java} model is mapped to plain \gls{XML}/\gls{XSD} with \gls{JAXB}, to a \gls{Hibernate} mapping file and to \gls{SQL}/\gls{DDL} statements.
Figure \ref{figure:ExampleCorpusJORXDomains} shows a schematic illustration of the linguistic domains involved:
\begin{description}

\item[Java]
The language and technology used to implement the \gls{101HRMS} model.

\item[XML]
The language used to serialize the \gls{101HRMS} model.

\item[SQL/DDL]
The language used to persist the \gls{101HRMS} model.

\item[JAXB]
The technology used to implement \gls{O/X-Mapping} of the \gls{101HRMS} model.

\item[Hibernate]
The technology used to implement \gls{O/R-Mapping} of the \gls{101HRMS} model.

\end{description}

\begin{figure}[h!]
\begin{center}
\includegraphics[width=.6\textwidth]{images/JORXDomains.png}
\end{center}
{
\scriptsize 
This schematic illustration depicts the interrelation among linguistic domains of example corpus used.
It depicts languages and technologies for \gls{O/R/X-Mapping} with \gls{Java}.
}
\caption{Example Corpus Domains: Java O/R/X}
\label{figure:ExampleCorpusJORXDomains}
\end{figure}

The languages (\gls{Java}, \gls{XML} \& \gls{SQL}) in Figure \ref{figure:ExampleCorpusJORXDomains} are displayed as disjoint square sets.
Technologies (\gls{JAXB} \& \gls{Hibernate}) are displayed as oval sets intersecting languages.
This is due to their linguistic nature, e.g. \gls{JAXB} produces specific \Gls{Java}- and \gls{XML}-Code which does not necessarily intersect with code produced by other technologies.
\gls{Hibernate} intersects all three languages.
It uses \gls{XML} files or \gls{Java}-Annotations for describing \gls{O/R-Mapping} of a data-model and generates \gls{SQL} artifacts according to that mapping.
In this sense, technologies create technology-specific subsets of a languages.

\section{Metrics}
\label{section:Metrics}

\section{Results}
\label{section:Results}

%\section{Link Proper Part Ratio}
%The ratio between all proper parts of two artifacts and the proper parts of the same artifacts in a relationship.
%
%\begin{align*}
%\pi_{R,A_1,A_2} = \frac{|\{ (p_1,p_2) \in R : p_1 \properPartOf A_1 \wedge p_2 \properPartOf A_2 \}|}{|\{ p : p \properPartOf A_1 \vee p \properPartOf A_2\}|}
%\end{align*}
\chapter{Conclusion}
\label{chapter:Conclusion}

\section{Future Work}

\paragraph*{Larger Corpus}
\paragraph*{Instance-Level Trace Link Recovery}
\paragraph*{Foreign Key Trace Link Recovery}

\clearpage
\printglossary

\clearpage
\chapter*{Addendum}
Source codes of this thesis, the implemented recovery system and its integration into \gls{MegaL/Xtext} is available on GitHub:
\begin{center}
\url{https://github.com/maxmeffert/BScThesis}
\end{center}

\noindent
Source code of the original implementation of \gls{MegaL/Xtext} is also available on GitHub:
\begin{center}
\url{https://github.com/avaranovich/megal-xtext}
\end{center}

%\clearpage
%\appendix
%\input{content/appendix.tex}

\cleardoublepage

%\bibliographystyle{acm}
\normalem % fix line breaks in book titles
\printbibliography

\end{document}


\section{Mereology}
Mereology is the logical study on the semantics of parthood.
\textit{"As a formal theory, mereology is simply
an attempt to set out the general principles underlying the relationships between a whole and its constituent parts [...]"} \cite{DBLP:journals/dke/Varzi96}.
Achille C. Varzi describes a collection of formal theories, i.e. sets of distinct axioms, of Mereology in \cite{DBLP:journals/dke/Varzi96}, which will be summarized in this section.
Note, that the terms relation, relationship and predicate may be used synonymously.

\subsection{Parthood}
First we define the intuitive notion of the parthood relationship:
\begin{definition}[$\partOf$]
Let $x$ and $y$ objects of interest.
We define:
\begin{align}
x \partOf y
:\Leftrightarrow
x \text{ is a constituent part of } y
\end{align}
\end{definition}
We further assume, that $\partOf$ satisfies the following properties:
\begin{align}
&\text{(P1)}
\qquad x \partOf x 
&\qquad \text{(Reflexivity)}
\\
&\text{(P2)}
\qquad x \partOf y \wedge y \partOf x \rightarrow x = y
&\qquad \text{(Antisymmetry)}
\\
&\text{(P3)}
\qquad x \partOf y \wedge y \partOf z \rightarrow x \partOf z
&\qquad \text{(Transitivity)}
\end{align}
Thus, $\partOf$ induces a partial order of things.

However, since the reflexive parthood may be to week for some cases, we also define a stricter, irreflexive parthood relationship as follows:
\begin{align}
x \properPartOf y
:\Leftrightarrow
x \partOf y \wedge \neg(y \partOf x)
\end{align}
Proper parthood induces a strict partial order of things.

In addition to the relationships above we introduce the following predicates in order to provide a more concise notation:
\begin{align}
x \overlaps y
&:\Leftrightarrow
\exists z : z \partOf x \wedge z \partOf y
&\qquad \text{(Overlap)}
\\
x \underlaps y
&:\Leftrightarrow
\exists z : x \partOf z \wedge y \partOf z
&\qquad \text{(Underlap)}
\end{align}
Overlap and Underlap model situations, where objects share a distinct part or belong to same distinct whole.

\subsection{Supplementation}


\subsection{Mereology Theories}

\chapter{Background}
%\lipsum[1]
TBD.

\section{Mathematical Background}
\subsection{Relations}
A $n$-ary, $n \in \mathbb{N}$, relation is the subset of the Cartesian product of $n$ sets:
\begin{align}
R \subseteq A_1 \times ... \times A_n 
= \{ (a_1,...,a_n) | a_1 \in A_1 \wedge ... \wedge a_n \in A_n \}
\end{align} 
and a binary relation is the subset of the Cartesian product between two sets:
\begin{align}
R \subseteq A \times B = \{ (a,b) | a \in A \wedge b \in B \}
\end{align}

\subsubsection{Homogeneous \& Heterogeneous Relations}
A relation $R \subseteq A \times B$ is called homogeneous if $A = B$, otherwise it is called heterogeneous.
However, an arbitrary relation $R \subseteq A \times B$ is also homogeneous in the sense of:
\begin{align}
R \subseteq A \times B \subseteq (A \cup B) \times (A \cup B)
\end{align}


\subsubsection{Composition}
The composition $S \circ R \subseteq A \times D$ of two binary relations $R \subseteq A \times B$ and $S \subseteq C \times D$ is defined as:
\begin{align}
S \circ R := \{ (a,d) \in A \times D | \exists x \in B \cap C : (a,x) \in R \wedge (x,d) \in S \}
\end{align}

\subsubsection{Identity Relation}
The relation $I$ over a set $A$ with:
\begin{align}
I &:= \{ (a,b) \in A \times A | a = b \} = \{ (a,a) | a \in A \} \subseteq A \times A
\end{align}
is called identity relation.

\subsubsection{Exponentiation}
The $n$-th power $R^n$ of a relation $R$ with $n \in \mathbb{N}$ is the product of its composition $n$-times with itself:
\begin{align}
R^n = \underbrace{R \circ ... \circ R}_{n}
\end{align}
and can be recursively defined with:
\begin{align}
R^{0} &:= I
\\R^{n} &:= R \circ R^{n-1}
\end{align}

\subsubsection{Properties}
A relation $R$ is called ...
\begin{center}
\begin{tabular}{ll}
\textbf{Property} & \textbf{Requirement}\\
bijective & $\forall b \in B \exists! a \in A: (a,b) \in R$\\
function & $\forall a \in A \exists! b \in B: (a,b) \in R$\\
reflexive & $\forall a \in A: (a,a) \in R$\\
irreflexive & $\forall a \in A: (a,a) \not\in R$\\
transitive & $\forall a,b,c \in A: (a,b) \in R \wedge (b,c) \in R \Rightarrow (a,c) \in R$\\
intransitive & $\forall a,b,c \in A: (a,b) \in R \wedge (b,c) \in R \Rightarrow (a,c) \not\in R$\\
symmetric & $\forall a,b \in A: (a,b) \in R \Rightarrow (b,a) \in R$\\
asymmetric & $\forall a,b \in A: (a,b) \in R \Rightarrow (b,a) \not\in R$\\
antisymmetric & $\forall a,b \in A: (a,b) \in R \wedge (b,a) \in R \Rightarrow a = b$\\
\end{tabular}
\end{center}

\subsubsection{Closures}
For a relation $R$ over a set $A$ its closure:
\begin{itemize}
\item
$R \cup I$
is called \textit{reflexive closure}

\item
$R^{+} := R^{1} \cup R^{2} \cup R^{3} \cup ...$
is called \textit{transitive closure}
\begin{align}
(a,b) \in R^{+} 
&\Leftrightarrow 
(a,b) \in R^{1} \vee (a,b) \in R^{2} \vee (a,b) \in R^{3} \vee ...
\\&\Leftrightarrow 
\exists n \in \mathbb{N} : n \geq 1 \wedge (a,b) \in R^{n}
\\&\Leftrightarrow 
\exists n \in \mathbb{N} : n \geq 1 \wedge (a,b) \in R \circ R^{n-1}
\\&\Leftrightarrow 
\exists n \in \mathbb{N} : n \geq 1 \wedge [\exists x \in A : (a,x) \in R \wedge (x,b) \in R^{n-1} ]
\\&\Leftrightarrow 
\exists n \in \mathbb{N} \exists x \in A : n \geq 1 \wedge (a,x) \in R \wedge (x,b) \in R^{n-1}
\end{align}

\item
$R^{*} := R^{+} \cup I$
is called \textit{reflexive-transitive closure}
\begin{align}
(a,b) \in R^{*}
&\Leftrightarrow 
(a,b) \in R^{+} \vee (a,b) \in I
\\&\Leftrightarrow 
(a,b) \in R^{+} \vee a = b
\end{align}
\end{itemize}



\section{Formal Languages \& Grammars}
\subsection{Context-Free Languages \& Grammars}

\section{Traceability}
\cite{Winkler:2010:STR:1861285.1861287}
%\lipsum[1]
TBD.
\subsection{Traceability Relationship}
\subsection{Traceability Link}
\subsection{Traceability Recovery}
\subsection{Traceability Exploration}

\section{Megamodeling}
%\lipsum[1]
TBD.

\subsection{\megal}
\subsubsection{\megalxtext}

\section{Ontologies}
%\lipsum[1]
TBD.

\section{Mereology}
\begin{align}
&x \partOf x 
&\qquad \text{(Reflexivity)}
\\&x \partOf y \wedge y \partOf x \Rightarrow x = y
&\qquad \text{(Antisymmetry)}
\\&x \partOf y \wedge y \partOf z \Rightarrow x \partOf z
&\qquad \text{(Transitivity)}
\end{align}
%\lipsum[1]
TBD.

\section{Program Analysis}
%\lipsum[1]
TBD.

\section{XML Data Binding}
%\lipsum[1]
TBD.


\subsection{Java Architecture for XML Binding (JAXB)}

\section{Object Relational Mapping}
%\lipsum[1]
TBD.



\subsection{Java Persistence API (JPA)}

\subsection{Hibernate}



\section{Another Tool For Language Recognition (ANTLR)}


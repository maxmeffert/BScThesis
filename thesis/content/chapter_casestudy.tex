\chapter{Mini Case-Study}
\label{chapter:MiniCaseStudy}
This chapter provides a mini case-study evaluating the developed recovery system for this thesis.
§\ref{section:ExampleCorpus} outlines the corpus used for evaluation.
§\ref{section:Metrics} describes metrics used to evaluate the system.
§\ref{section:Results} discusses results of the case-study.

\section{Corpus}
\label{section:ExampleCorpus}
The example corpus used to develop the recovery system for this thesis consists of artifacts implementing a fictional \gls{HRMS} within an \gls{O/R/X-Mapping} scenario using \gls{Java} technologies.
The model is provided by the 101wiki\footnote{\url{https://101wiki.softlang.org/} (retrieved \formatdate{12}{11}{2017})}, where it is used for contributions.
It is implemented using plain \Gls{Java} and then mapped to plain \gls{XML}/\gls{XSD} with \gls{JAXB} and to \gls{SQL/DDL} statements using \gls{Hibernate} mapping files and/or annotations.



\subsection{The 101HRMS Model}
\gls{101HRMS}\footnote{\url{https://101wiki.softlang.org/101:@system} (retrieved \formatdate{12}{11}{2017})} provides a simple model of a company with many departments and employees.
Figure \ref{figure:101HRMSModel} shows an \gls{UML} class diagram of a variant of this model.

\begin{figure}[h!]
\begin{center}
\includegraphics[scale=.5]{images/101HRMSModel.png}
\end{center}
{
\scriptsize 
This \gls{UML} class diagram depicts the model of the \gls{101HRMS}.
It consists of simple companies with nested departments and employees mapped to the latter.
}
\caption{The 101 Human Resource Management System Model}
\label{figure:101HRMSModel}
\end{figure}

The \gls{101HRMS} model consists of companies attributed with a name.
Each company accumulates departments.
Each department is also attributed with a name, aggregates employees and has one employee acting as manager.
Departments can further be refined into sub-departments.
Each employee is attributed with a name, an age and a salary.
Each entity is also attributed with an ID.

\subsection{Linguistic Domains of the Example Corpus}
The example corpus used to develop the recovery system contains artifacts implementing the \gls{101HRMS} model generated or used by \gls{Java} technologies for \gls{O/R/X-Mapping}, i.e. a \gls{Java} model is mapped to plain \gls{XML}/\gls{XSD} with \gls{JAXB}, to a \gls{Hibernate} mapping file and to \gls{SQL}/\gls{DDL} statements.
Figure \ref{figure:ExampleCorpusJORXDomains} shows a schematic illustration of the linguistic domains involved:
\begin{description}

\item[Java]
The language and technology used to implement the \gls{101HRMS} model.

\item[XML]
The language used to serialize the \gls{101HRMS} model.

\item[SQL/DDL]
The language used to persist the \gls{101HRMS} model.

\item[JAXB]
The technology used to implement \gls{O/X-Mapping} of the \gls{101HRMS} model.

\item[Hibernate]
The technology used to implement \gls{O/R-Mapping} of the \gls{101HRMS} model.

\end{description}

\begin{figure}[h!]
\begin{center}
\includegraphics[width=.6\textwidth]{images/JORXDomains.png}
\end{center}
{
\scriptsize 
This schematic illustration depicts the interrelation among linguistic domains of example corpus used.
It depicts languages and technologies for \gls{O/R/X-Mapping} with \gls{Java}.
}
\caption{Example Corpus Domains: Java O/R/X}
\label{figure:ExampleCorpusJORXDomains}
\end{figure}

The languages (\gls{Java}, \gls{XML} \& \gls{SQL}) in Figure \ref{figure:ExampleCorpusJORXDomains} are displayed as disjoint square sets.
Technologies (\gls{JAXB} \& \gls{Hibernate}) are displayed as oval sets intersecting languages.
This is due to their linguistic nature, e.g. \gls{JAXB} produces specific \Gls{Java}- and \gls{XML}-Code which does not necessarily intersect with code produced by other technologies.
\gls{Hibernate} intersects all three languages.
It uses \gls{XML} files or \gls{Java}-Annotations for describing \gls{O/R-Mapping} of a data-model and generates \gls{SQL} artifacts according to that mapping.
In this sense, technologies create technology-specific subsets of a languages.

\section{Metrics}
\label{section:Metrics}

\section{Results}
\label{section:Results}

%\section{Link Proper Part Ratio}
%The ratio between all proper parts of two artifacts and the proper parts of the same artifacts in a relationship.
%
%\begin{align*}
%\pi_{R,A_1,A_2} = \frac{|\{ (p_1,p_2) \in R : p_1 \properPartOf A_1 \wedge p_2 \properPartOf A_2 \}|}{|\{ p : p \properPartOf A_1 \vee p \properPartOf A_2\}|}
%\end{align*}
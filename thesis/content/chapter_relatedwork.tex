\chapter{Related Work}
\label{chapter:RelatedWork}
This chapter provides an non-exhaustive overview over related work regarding trace link recovery approaches.

%\section{Linguistic Architectures}
%
%\paragraph*{Modeling the Linguistic Architecture of Software Products}
%\cite{DBLP:conf/models/FavreLV12}
%
%
%\section{Trace Recovery}

\section{Paper Summary}

\paragraph*{Information Retrieval Methods for Automated Traceability Recovery}
\cite{DeLucia2012} 
proposes a semi-automatic approach for recovering trace links:
\begin{enumerate}
\item
corpus generation using document parsing to extract \glspl{Artifact} of a desired size (paragraphs, classes, methods, etc.)
\item
corpus indexing using information retrieval methods in order to create a homogeneous corpus removing stop words and applying other normalization techniques
\item
recovery of "candidate" links computing a probabilistic similarity
\item
analysis/assessment of candidate links conducted by a human removing false positive links and confirming correct links
\end{enumerate}
It is argued that using probabilistic information retrieval methods reduces the  overhead created by development of language specific analysis solutions, e.g. \gls{AST} generation.


\paragraph*{Using Rules for Traceability Creation} 
\cite{Zisman2012}
proposes a fully automatic approach for recovering trace links by normalizing to \gls{XML} documents and applying XQuery encoded traceability rules on the corpus.
This approach also takes ambiguity of natural languages into account by applying a grammatical classification of sentences during preprocessing and considering synonyms during recovery.

\paragraph*{Tracing object-oriented code into functional requirements} 
\cite{AntoniolCCDM00}
proposes an information retrieval approach for recovering trace links between source code and natural language documentation \glspl{Artifact}.
In a provided cast-sudy, \gls{Java} classes are linked to requirement documents based on a probabilistic similarity computed from the set of identifiers found in a class \gls{Artifact}.

\paragraph*{Recovering Code to Documentation Links in OO Systems}
\cite{AntoniolCDLM99}
proposes an information retrieval approach for recovering trace links between source code and natural language documentation \glspl{Artifact} based on a  probabilistic similarity computed from the set of identifiers found in source code \glspl{Artifact}.
This is a predecessor paper of \cite{AntoniolCCDM00}.

\paragraph*{Mining software repositories for traceability links}
\cite{KagdiMS07}
proposes an approach for recovering trace links among source code and informal documentation \glspl{Artifact} by monitoring their co-change frequency, i.e. \glspl{Artifact} that are changed to concurrently in the history of a version control system.
It is argued that a high co-change frequency implies trace links among \glspl{Artifact}.

%\paragraph*{CaCOphoNy: metamodel-driven software architecture reconstruction}
%\cite{Favre2004}

\paragraph*{Combining Textual and Structural Analysis of Software Artifacts for  Traceability Link Recovery}
\cite{McMillanPR2009}
proposes a combined approach of information retrieval methods for recovering exogenous links between source code and documentation \glspl{Artifact} and classical analysis methods for recovering endogenous links among source code \glspl{Artifact}.
Endogenous links among documentation \glspl{Artifact} are proposed to be recovered indirectly, i.e. if source \glspl{Artifact} $S1$ and $S2$ are linked and both ar linked to documentation \glspl{Artifact} $D1$ and $D2$ respectively, then this may be seen as evidence that $D1$ are also linked $D2$.
Indirect linking is proposed because the authors argue that information retrieval methods alone are not sufficient for trace link recovery purposes.
Documentation \glspl{Artifact} may be linked despite sharing little to no semantic similarities.

\paragraph*{Traceability Management for Impact Analysis}
\cite{DelukaFR2008}

\paragraph*{Automating traceability link recovery through classification}
\cite{Mills:2017:ATL:3106237.3121280}

\paragraph*{Recovering test-to-code traceability using slicing and textual analysis} 
\cite{Qusef:2014:RTT:2747015.2747194}

\paragraph*{Recovering traceability links in software artifact management systems using information retrieval methods}
\cite{Lucia:2007:RTL:1276933.1276934}

\paragraph*{Semantic driven program analysis}
\cite{Marcus2004}


\section{Remarks}



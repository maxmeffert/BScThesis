\chapter{Hypotheses}
\label{chapter:Hypotheses}
This chapter introduces the hypotheses this thesis is bases upon.
TBD.

\section{Fragments}
\label{section:Fragments}
The Oxford English  Dictionary defines fragment\footnote{\url{https://en.oxforddictionaries.com/definition/fragment} (retrieved 08/28/2017)} as \citedtext{[...] isolated or incomplete part of something [...]}.
Regarding natural languages, sentence fragments are incomplete sentences lacking one or more grammatical elements. \ToDo{Lacks citation for "sentence fragment"!}

The intuitive notion of a fragment when working with and/or talking over computer programs is, that it is a piece of code containing a functionality of momentary interest.
In that regard, we exclusively use the word fragment referring to isolated or incomplete parts of things having a textual representation, e.g. computer programs.
However, computer science knows different levels of organization in text, which effects the semantics of textual fragments.

The most basic level of textual organization is no organization at all.
Random text is only restricted by its alphabet, which allows us to specify fragmentation solely in terms of concatenation.

\begin{definition}[Fragments over Alphabets]
\label{definition:FragmentsOverAlphabets}
Let $\Sigma$ be an alphabet and $x,y \in \Sigma^\ast$ be two words over that alphabet.
$x$ is a fragment of $y$, if and only if there is a concatenation containing $x$ and resulting in $y$:
\begin{align}
&\fragmentOf : \Sigma^\ast \times \Sigma^\ast
\\
&x \fragmentOf y
:\Leftrightarrow
\exists u,v \in \Sigma^\ast (u x v = y)
\end{align}
\end{definition}

We can also observe from definition \ref{definition:FragmentsOverAlphabets} that textual fragmentation shares similarities with mereological parthood, i.e. the $\fragmentOf$ definition is also reflexive, antisymmetric and transitive:
\begin{enumerate}
\item
Reflexivity: $x \fragmentOf x$\\
$x \fragmentOf x$ holds for all $x$ since $u = v = \epsilon \in \Sigma^\ast$.
\item
Antisymmetry: $x \fragmentOf y \wedge y \fragmentOf x \rightarrow x = y$\\
$x \fragmentOf y \wedge y \fragmentOf x$ is equivalent to $axb = y \wedge cyd = x$, from which follows $caxbd = x$ and then $a=b=c=d=\epsilon$, thus $x = y$ follows eventually
\item
Transitivity: $x \fragmentOf y \wedge y \fragmentOf z \rightarrow x \fragmentOf z$\\
$x \fragmentOf y \wedge y \fragmentOf z$ is equivalent to $axb = y \wedge cyd = z$, from which follows $caxbd = z$, which is in turn equivalent to $uxv = z$ with $u = ca$ and $v = bd$, thus $x \fragmentOf z$ follows eventually
\end{enumerate}
There is a universal bottom element, which is fragment of all words:
\begin{align}
\forall x \in \Sigma^\ast (\epsilon \fragmentOf x)
\end{align}
and there is a universal top element, all words are fragments of:
\begin{align}
\begin{split}
&\forall x \in \Sigma^\ast (x \fragmentOf W)
\\
&W := w_1 w_2 w_3 ... \qquad w_i \in \Sigma^\ast \backslash \{\epsilon\}, i \in \mathbb{N}
\end{split}
\end{align}
where $W$ is the concatenation of all words in $\Sigma^\ast$ and thus is an infinite word. $W \in \Sigma^\omega$ is interpreted as mapping $\mathbb{N} \rightarrow \Sigma$, giving a symbol from the alphabet for its position in the word. \ToDo{citation for infinite words, Handbook on Formal languages}

We can now easily agree upon, that $\fragmentOf$ is $\partOf$ for texts.
Thus both are logically equivalent relations:
\begin{align}
\fragmentOf \equiv \partOf
\end{align}
\ToDo{refer to philosophical problems from SEP}

A higher level of textual organization are formal languages, i.e. subsets of $\Sigma^\ast$ for any alphabet $\Sigma$ \cite{DBLP:books/daglib/0090590}.
Given any language specified with set-builder notation:
\begin{align}
L := \{ w \in \Sigma^\ast | \Phi(w) \} \subseteq \Sigma^\ast 
\end{align}
where $\Phi(w)$ is an arbitrary restriction on $w$, we can still observe the same behavior regarding fragments, e.g.:
\begin{align*}
&L = \{ a^n b^n | n \in \mathbb{N} : n \geq 2 \} \subseteq \{a,b\}^\ast 
= \{ aabb, aaabbb, aaaabbbb, ... \}
\\&...
\\&ab \fragmentOf aabb
\\&aabb \fragmentOf aaabbb
\\&aaabbb \fragmentOf aaaabbbb
\\&...
\end{align*}
However, we can only observe the behavior partly for $\fragmentOf: \Sigma^\ast \times \Sigma^\ast$ and $\fragmentOf: \Sigma^\ast \times L$, but not for $\fragmentOf: L \times L$.
In other words, the fragments of words in formal languages are not necessarily words of that language itself.
This becomes even more apparent if we consider fragments and languages generated by formal grammars.

\begin{definition}[Fragments generated by Grammars]
Let $G = (V,\Sigma,P,S)$ be a grammar and $x,y \in \Sigma^\ast$ be two words over its terminal symbols. $x$ is a fragment of $y$ generated by $G$, if and only if there is a derivation from a concatenation containing $x$ to $y$:
\begin{align}
&x \fragmentOfG{G} y 
\Leftrightarrow
\exists u,v \in (V \cup \Sigma)^\ast 
(S \Rightarrow_G^\ast u x v \Rightarrow_G^\ast y)
\end{align}
\end{definition}



\subsection{Fragment Languages}

\section{Correspondence}
The Oxford English Dictionary defines correspondence\footnote{\url{https://en.oxforddictionaries.com/definition/correspondence} (retrieved 08/28/2017)} as \citedtext{[...] close similarity, connection, or equivalence [...]}.
\cite{DBLP:conf/sle/Lammel16}

\begin{definition}[Correspondence]
Let $R \subseteq L_1 \times L_2$ be a relation between two languages $L_1$ and $L_2$.
Two artifacts $a_1$ and $a_1$ recursively correspond to each other in the sense of $R$ if both artifacts contain parts, which correspond in the same fashion to exactly one part of the other artifact:
\begin{align}
\begin{split}
&(a_1,a_2) \in R
\\&\wedge \forall b_1 \in L_1 \exists! b_2 \in L_2 
(b_1 \partOf a_1 \rightarrow (b_2 \partOf a_2 \wedge b_1 \correspondsToR{R} b_2))
\\&\wedge \forall b_2 \in L_2 \exists! b_1 \in L_1 
(b_2 \partOf a_2 \rightarrow ( b_1 \partOf a_2 \wedge b_2 \correspondsToR{R} b_1))
\\&\Rightarrow a_1 \correspondsToR{R} a_2
\end{split}
\end{align}
\end{definition}




\section{Conformance}
The Oxford English Dictionary defines conformance\footnote{\url{https://en.oxforddictionaries.com/definition/conformance} (retrieved 08/28/2017)} as synonym for conformity\footnote{\url{https://en.oxforddictionaries.com/definition/conformity} (retrieved 08/28/2017)} and thus in turn as \citedtext{Compliance with standards, rules, or laws}.

\cite{DBLP:conf/sle/Lammel16}
\begin{definition}[Conformance]
Let $L \subseteq \Any$ be a language, $\DefL{L} \subseteq \Any$ the definition language of $L$ and $\textsf{def}_L \in \DefL{L}$ the actual definition artifact of $L$.
An arbitrary artifact conforms to the definition artifact of $L$, if and only if it is an element of $L$:
\begin{align}
\forall x \in \Any 
(x \conformsTo \textsf{def}_L \Leftrightarrow x \in L)
\end{align}
\end{definition}

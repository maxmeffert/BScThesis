\section{Relations}
This section introduces the necessary aspects of mathematical relations for this thesis.
The concept of relations is a generalization of semantic dependencies between two or more mathematical objects.
This section is based on \cite{DBLP:books/sp/SchmidtS89}.

Relations are based on set-theory. We also introduce the necessary constructs of set-theory in order to clarify terminology and notation.
A set is a collection of well distinguishable mathematical objects.
Objects in a set are called elements of the set.
A set does not contain two or more identical elements.
The notation $x \in X$ denotes that $x$ is an element of the set $X$.
The symbol $\emptyset$ denotes the \emph{empty set}, which contains no elements.
The symbol $\universe$ denotes the universal set, which contains all elements.

\begin{definition}[Inclusion]
\label{definition:Inclusion}
Let $X$ and $Y$ be a sets.
$Y$ \emph{includes} $X$ if and only if:
\begin{align}
X \subset Y :\Leftrightarrow \forall x [x \in X \Rightarrow x \in Y]
\end{align}
Then $X$ is called \emph{subset} of $Y$ and $Y$ is called \emph{superset} of $X$.
For an arbitrary set $Z \neq \emptyset$, the statement $\emptyset \subset Z$ is always true, respectively $Z \subset \emptyset$ is always false.
\end{definition}

\begin{definition}[Union]
Let $X$ and $Y$ be a sets.
\begin{align}
X \cup Y &:= \{ x | x \in X \vee x \in Y \} 
\end{align}
$X \cup Y$ is called \emph{union} of $X$ and $Y$.
\end{definition}

\begin{definition}[Intersection]
Let $X$ and $Y$ be a sets.
\begin{align}
X \cap Y &:= \{ x | x \in X \wedge x \in Y \} 
\end{align}
$X \cap Y$ is called \emph{intersection} of $X$ and $Y$.
\end{definition}

\begin{definition}[Power-Set]
\label{definition:PowerSet}
Let $X$ be a set, then the \emph{power-set} of $X$ is defined as:
\begin{align}
\mathcal{P}(X) := \{ Y | Y \subset X \}
\end{align}
\end{definition}

The definition of inclusion provides an order for power-sets.
So we may compare two sets $A$ and $B$ in the sense of \emph{smaller} and \emph{larger}, i.e.:
\begin{align}
A \text{ is smaller than } B 
&\Leftrightarrow A \subset B
\\
A \text{ is larger than } B  
&\Leftrightarrow B \subset A
\\
A \text{ is the smallest subset of } B
&\Leftrightarrow \forall C \in \powersetOf{B} : A \subset C
\\
A \text{ is the largest subset of } B
&\Leftrightarrow \forall C \in \powersetOf{B} : C \subset A
\end{align}


\begin{definition}[Upper \& Lower Bound]
Let $\universe$ be a universe, $X \in \powersetOfUniverse$ be a set in the universe and $A \subset \mathcal{P}(U), A \neq \emptyset,$ non-empty subsets in the universe.
\begin{align}
X \text{ is an \emph{upper bound} for } A
&:\Leftrightarrow
\forall Y \in A : Y \subset X
\\
X \text{ is a \emph{lower bound} for } A
&:\Leftrightarrow
\forall Y \in A : X \subset Y
\end{align}
We also define:
\begin{align}
\mathbf{U}_A 
&:= \{ U \in \powersetOfUniverse | \forall Y \in A : Y \subset U \}
\\
\mathbf{L}_A
&:= \{ L \in \powersetOfUniverse | \forall Y \in A : L \subset Y \}
\end{align}
as sets of all upper/lower bounds for $A$.
\end{definition}

Because our definition of upper and lower bounds is based on power-sets, existence is guaranteed:
Given an arbitrary set $S$, the $S$ and $\emptyset$ are always elements of $\powersetOf{S}$.
For each element $Y$ of a non-empty selection $A \subset \powersetOf{S}$ of the power-set, $Y \subset S$ and $\emptyset \subset Y$ holds.
So $S$ is an upper and $\emptyset$ is a lower bound for $A$.

\begin{definition}[Supremum \& Infimum]
\label{definition:SupremumAndInfimum}
Let $\universe$ be a universe, $X \in \powersetOfUniverse$ be sets in the universe and $A \subset \powersetOfUniverse, A \neq \emptyset$ a non-empty selection of sets in the universe.
If
\begin{align}
X 
= \sup A
:= \bigcup\limits_{Y \in A} Y
&:\Leftrightarrow
X \in \mathbf{U}_A \wedge \forall U \in \mathbf{U}_A : X \subset U
\\
X
= \inf A
:= \bigcap\limits_{Y \in A} Y
&:\Leftrightarrow
X \in \mathbf{L}_A \wedge \forall L \in \mathbf{L}_A : L \subset X
\end{align}
then $X$ is called \emph{supremum}/\emph{infimum} for $A$.
\end{definition}

Existence for supremum and infimum is guaranteed, because upper and lower bounds exist as shown above.
Thus, for any non-empty selection $A \subset \powersetOf{S}$ of a power-set, $\mathbf{U}_A$ and $\mathbf{L}_A$ are not empty.
So we need to proof, that $X = \bigcup\limits_{Y \in A} Y$ respectively $X = \bigcap\limits_{Y \in A} Y$ are in fact the smallest upper and the  largest lower bound.
Or in other words: Is there another bound $X' \in \mathbf{U}_A$ or $X' \in \mathbf{L}_A$ with $X' \neq X$ so that $X' \subset X$ respectively $X \subset X'$ holds?
\begin{enumerate}
\item
Supremum:
We assume $X' \in \mathbf{U}_A$ with $X' \neq X$ and $X' \subset X$ for $X = \bigcup\limits_{Y \in A} Y$ exists, then an element $x \in X$ exists, which is not element of $X'$.
Because $X$ is the union of all sets in selection $A$, $x$ must be element of at least one of its sets.
However, then $X'$ cannot include sets containing $x$.
Thus, $X'$ cannot be an upper bound for $A$ and $X = \sup A$.

\item
Infimum:
We assume $X' \in \mathbf{L}_A$ with $X' \neq X$ and $X \subset X'$ for $X = \bigcap\limits_{Y \in A} Y$ exists, then an element $x \in X'$ exists, which is not element of $X$.
Because $X$ is the intersection of all sets in selection $A$, $x$ cannot be element of at least one of its sets.
However, then $X'$ must include sets containing $x$.
Thus, $X'$ cannot be a lower bound fo $A$ and $X = \inf A$.

\end{enumerate}
Supremum and Infimum are unique for any non-empty selection of sets in a universe and can be obtained by its union respectively its intersection.



\begin{definition}[Cartesian Product]
Let $U$ be a universe and $X_n \in \mathcal{P}(U)$ sets with $ i=1...n, n \in \mathbb{N}$, then:
\begin{align}
X_1 \times ... \times X_n := \{ (x_1,..., x_n) \}
\end{align} 
is called \emph{Cartesian product}.
\end{definition}

\begin{definition}[Relation]
A relation is a subset of a Cartesian product:
\begin{align}
R \subset X_1 \times ... \times X_n 
\end{align}
The relation of only two sets is called \emph{binary relation}.
Instead of writing $(x,y) \in R$ we may also use the shorter notation $xRy$.
\end{definition}

An arbitrary relation $R \subset A \times B$ is called \emph{homogeneous} if $A = B$, otherwise it is called \emph{heterogeneous}.
However, an arbitrary relation $R \subset A \times B$ is also homogeneous in the sense of
$R \subset A \times B \subset (A \cup B) \times (A \cup B)$.

\begin{definition}[Properties of Relations]
Let $R \subset A \times A$ and $S \subset A \times B$ be homogeneous or arbitrary relations, they may satisfy one or more of the following properties:
\begin{align}
\emph{bijective} 
&:\Leftrightarrow
\forall b \in B \exists! a \in A: (a,b) \in S
\\
\emph{function} 
&:\Leftrightarrow
\forall a \in A \exists! b \in B: (a,b) \in S
\\
\emph{reflexive} 
&:\Leftrightarrow
\forall a \in A: (a,a) \in R
\\
\emph{irreflexive} 
&:\Leftrightarrow
\forall a \in A: (a,a) \not\in R
\\
\emph{transitive} 
&:\Leftrightarrow
\forall a,b,c \in A: (a,b) \in R \wedge (b,c) \in R \Rightarrow (a,c) \in R
\\
\emph{intransitive} 
&:\Leftrightarrow
\forall a,b,c \in A: (a,b) \in R \wedge (b,c) \in R \Rightarrow (a,c) \not\in R
\\
\emph{symmetric} 
&:\Leftrightarrow
\forall a,b \in A: (a,b) \in R \Rightarrow (b,a) \in R
\\
\emph{asymmetric} 
&:\Leftrightarrow
\forall a,b \in A: (a,b) \in R \Rightarrow (b,a) \not\in R
\\
\emph{antisymmetric} 
&:\Leftrightarrow
\forall a,b \in A: (a,b) \in R \wedge (b,a) \in R \Rightarrow a = b
\end{align}
\end{definition}

\begin{definition}[Reflexive Closure]
Let $R$ be a homogeneous relation.
\begin{align}
R^\circ
:= \inf \{ C | C \supset R \wedge C \text{ reflexive } \}
= R \cup I
\end{align}
\end{definition}


\begin{definition}[Composition of Binary Relations]
The composition $S \circ R \subseteq A \times D$ of two binary relations $R \subseteq A \times B$ and $S \subseteq C \times D$ is defined as:
\begin{align}
S \circ R := \{ (a,d) \in A \times D | \exists x \in B \cap C : (a,x) \in R \wedge (x,d) \in S \}
\end{align}
\end{definition}

\begin{definition}[Identity Relation]
The relation $I$ over a set $A$ with:
\begin{align}
I &:= \{ (a,b) \in A \times A | a = b \} = \{ (a,a) | a \in A \} \subseteq A \times A
\end{align}
is called identity relation.
\end{definition}

The $n$-th power $R^n$ of a relation $R$ with $n \in \mathbb{N}$ is the product of its composition $n$-times with itself:
\begin{align}
R^n = \underbrace{R \circ ... \circ R}_{n}
\end{align}
and can be recursively defined with:
\begin{align}
R^{0} &:= I
\\R^{n} &:= R \circ R^{n-1}
\end{align}


\subsection{Closures}
For a relation $R$ over a set $A$ its closure:
\begin{itemize}
\item
$R \cup I$
is called \textit{reflexive closure}

\item
$R^{+} := R^{1} \cup R^{2} \cup R^{3} \cup ...$
is called \textit{transitive closure}
\begin{align}
(a,b) \in R^{+} 
&\Leftrightarrow 
(a,b) \in R^{1} \vee (a,b) \in R^{2} \vee (a,b) \in R^{3} \vee ...
\\&\Leftrightarrow 
\exists n \in \mathbb{N} : n \geq 1 \wedge (a,b) \in R^{n}
\\&\Leftrightarrow 
\exists n \in \mathbb{N} : n \geq 1 \wedge (a,b) \in R \circ R^{n-1}
\\&\Leftrightarrow 
\exists n \in \mathbb{N} : n \geq 1 \wedge [\exists x \in A : (a,x) \in R \wedge (x,b) \in R^{n-1} ]
\\&\Leftrightarrow 
\exists n \in \mathbb{N} \exists x \in A : n \geq 1 \wedge (a,x) \in R \wedge (x,b) \in R^{n-1}
\end{align}

\item
$R^{*} := R^{+} \cup I$
is called \textit{reflexive-transitive closure}
\begin{align}
(a,b) \in R^{*}
&\Leftrightarrow 
(a,b) \in R^{+} \vee (a,b) \in I
\\&\Leftrightarrow 
(a,b) \in R^{+} \vee a = b
\end{align}
\end{itemize}
